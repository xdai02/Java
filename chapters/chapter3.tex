\chapter{判断}

\section{逻辑运算符}

\subsection{关系运算符}

\begin{table}[H]
	\centering
	\setlength{\tabcolsep}{5mm}{
		\begin{tabular}{|c|c|}
			\hline
			\textbf{数学符号} & \textbf{关系运算符} \\
			\hline
			$ < $             & <                   \\
			\hline
			$ > $             & >                   \\
			\hline
			$ \le $           & <=                  \\
			\hline
			$ \ge $           & >=                  \\
			\hline
			$ \ne $           & !=                  \\
			\hline
			$ = $             & ==                  \\
			\hline
		\end{tabular}
	}
	\caption{关系运算符}
\end{table}

\subsection{逻辑运算符}

Java中逻辑运算符有三种:

\begin{enumerate}
	\item 逻辑与\&\&(logical AND):当多个条件同时为真,结果为真。
	      \begin{table}[H]
		      \centering
		      \setlength{\tabcolsep}{5mm}{
			      \begin{tabular}{|c|c|c|}
				      \hline
				      \textbf{条件1} & \textbf{条件2} & \textbf{条件1 \&\& 条件2} \\
				      \hline
				      T              & T              & T                         \\
				      \hline
				      T              & F              & F                         \\
				      \hline
				      F              & T              & F                         \\
				      \hline
				      F              & F              & F                         \\
				      \hline
			      \end{tabular}
		      }
		      \caption{逻辑与}
	      \end{table}

	\item 逻辑或||(logical OR):多个条件有一个为真时,结果为真。
	      \begin{table}[H]
		      \centering
		      \setlength{\tabcolsep}{5mm}{
			      \begin{tabular}{|c|c|c|}
				      \hline
				      \textbf{条件1} & \textbf{条件2} & \textbf{条件1 || 条件2} \\
				      \hline
				      T              & T              & T                       \\
				      \hline
				      T              & F              & T                       \\
				      \hline
				      F              & T              & T                       \\
				      \hline
				      F              & F              & F                       \\
				      \hline
			      \end{tabular}
		      }
		      \caption{逻辑或}
	      \end{table}

	\item 逻辑非!(logical NOT):条件为真时,结果为假;条件为假时,结果为真。
	      \begin{table}[H]
		      \centering
		      \setlength{\tabcolsep}{5mm}{
			      \begin{tabular}{|c|c|}
				      \hline
				      \textbf{条件} & \textbf{!条件} \\
				      \hline
				      T             & F              \\
				      \hline
				      F             & T              \\
				      \hline
			      \end{tabular}
		      }
		      \caption{逻辑非}
	      \end{table}
\end{enumerate}

\newpage

\section{if}

\subsection{if}

当if语句的条件为真时,进入花括号执行内部的代码;若条件为假,则跳过花括号执行后面的代码。 \\

if语句主要有以下几种形式: