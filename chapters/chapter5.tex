\chapter{数组}

\section{一维数组}

\subsection{数组(Array)}

一个变量只能存储一个内容,如果需要存储更多数据,就需要使用数组解决问题。一个数组变量可以存放多个数据,数组是一个值的集合,它们共享同一个名字,数组中的每个变量都能被其下标所访问。

\vspace{-0.5cm}

\begin{lstlisting}[language=Java]
int[] number = new int[10];
float[] grade = new float[50];
\end{lstlisting}

\begin{figure}[H]
	\centering
	\begin{tikzpicture}[scale=0.5]
		\draw[-] (0,0) -- (5,0) -- (10,0) -- (15,0) -- (20,0) -- (25,0) -- (25,3) -- (20,3) -- (15,3) -- (10,3) -- (5,3) -- (0,3) -- (0,0);
		\draw[-] (5,0) -- (5,3);
		\draw[-] (10,0) -- (10,3);
		\draw[-] (15,0) -- (15,3);
		\draw[-] (20,0) -- (20,3);

		\draw (2.5,1.5) node {a[0]};
		\draw (7.5,1.5) node {a[1]};
		\draw (12.5,1.5) node {a[2]};
		\draw (17.5,1.5) node {a[3]};
		\draw (22.5,1.5) node {a[4]};
	\end{tikzpicture}
\end{figure}

\begin{itemize}
	\item 元素:数组中的每个变量
	\item 大小:数组的容量
	\item 下标 / 索引(index):元素的位置,下标从0开始,必须为非负整数
\end{itemize}

\subsection{数组初始化}

一维数组可以在声明时进行初始化:

\vspace{-0.5cm}

\begin{lstlisting}[language=Java]
int[] arr = {3, 6, 8, 2, 4, 0, 9, 7, 1, 5};
int[] arr = new int[] {3, 6, 8, 2, 4, 0, 9, 7, 1, 5};
\end{lstlisting}

很多时候在使用数组之前需要将数组的内容全部清空,这可以利用循环来实现。 \\

\mybox{一维数组初始化}

\begin{lstlisting}[language=Java]
public class InitArr {
    public static void main(String[] args) {
        int[] arr = new int[100];
        for(int i = 0; i < arr.length; i++) {
            arr[i] = 0;
        }
    }
}
\end{lstlisting}

\vspace{0.5cm}

\mybox{数组最大值和最小值}

\begin{lstlisting}[language=Java]
public class MaxMin {
    public static void main(String[] args) {
        int[] num = {7, 6, 2, 9, 3, 1, 4, 0, 5, 8};
        int max = num[0];
        int min = num[0];

        for(int i = 1; i < num.length; i++) {
            if(num[i] > max) {
                max = num[i];
            } else if(num[i] < min) {
                min = num[i];
            }
        }

        System.out.println("max = " + max);
        System.out.println("min = " + min);
    }
}
\end{lstlisting}

\begin{tcolorbox}
	\mybox{运行结果}
	\begin{verbatim}
max = 9
min = 0
	\end{verbatim}
\end{tcolorbox}

\subsection{for-each}

for-each环是for循环的特殊简化版。

\vspace{-0.5cm}

\begin{lstlisting}[language=Java]
for(dataType var : set) {
    // code
}
\end{lstlisting}

\vspace{0.5cm}

\mybox{遍历数组}

\begin{lstlisting}[language=Java]
public class ForEach {
    public static void main(String[] args) {
        int[] arr = {7, 6, 2, 9, 3, 1, 4, 0, 5, 8};
        for(int elem : arr) {
            System.out.print(elem + " ");
        }
    }
}
\end{lstlisting}

\begin{tcolorbox}
	\mybox{运行结果}
	\begin{verbatim}
7 6 2 9 3 1 4 0 5 8
	\end{verbatim}
\end{tcolorbox}

\newpage

\section{二维数组}

\subsection{二维数组(2D Array)}

二维数组包括行和列两个维度,可以看成是由多个一维数组组成。

\begin{table}[H]
	\centering
	\setlength{\tabcolsep}{5mm}{
		\begin{tabular}{|c|c|c|c|}
			\hline
			a[0][0] & a[0][1] & a[0][2] & a[0][3] \\
			\hline
			a[1][0] & a[1][1] & a[1][2] & a[1][3] \\
			\hline
			a[2][0] & a[2][1] & a[2][2] & a[2][3] \\
			\hline
		\end{tabular}
	}
\end{table}

二维数组可以在声明时进行初始化:

\vspace{-0.5cm}

\begin{lstlisting}[language=Java]
int[][] arr = new int[2][3];
int[][] arr = {{1, 2, 3}, {4, 5, 6}};
\end{lstlisting}

\vspace{0.5cm}

\mybox{初始化二维数组}

\begin{lstlisting}[language=Java]
public class Init2dArr {
    public static void main(String[] args) {
        int[][] arr = new int[3][4];
        for(int i = 0; i < arr.length; i++) {
            for(int j = 0; j < arr[i].length; j++) {
                arr[i][j] = 0;
            }
        }
    }
}
\end{lstlisting}

\vspace{0.5cm}

\mybox{矩阵运算}

\vspace{0.5cm}

矩阵的加法/减法是指两个矩阵把其相对应元素进行加减的运算。 \\

矩阵加法:两个$ m \times n $矩阵A和B的和,标记为$ A + B $,结果为一个$ m \times n $的矩阵,其内的各元素为其相对应元素相加后的值。 \\

矩阵减法:两个$ m \times n $矩阵A和B的差,标记为$ A - B $,结果为一个$ m \times n $的矩阵,其内的各元素为其相对应元素相减后的值。

\begin{align}\nonumber
	\left[\begin{matrix}
			1 & 3 \\
			1 & 0 \\
			1 & 2 \\
		\end{matrix} \right]
	+
	\left[\begin{matrix}
			0 & 0 \\
			7 & 5 \\
			2 & 1 \\
		\end{matrix} \right]
	=
	\left[\begin{matrix}
			1+0 & 3+0 \\
			1+7 & 0+5 \\
			1+2 & 2+1 \\
		\end{matrix} \right]
	=
	\left[\begin{matrix}
			1 & 3 \\
			8 & 5 \\
			3 & 3 \\
		\end{matrix} \right]
\end{align}

\begin{align}\nonumber
	\left[\begin{matrix}
			1 & 3 \\
			1 & 0 \\
			1 & 2 \\
		\end{matrix} \right]
	-
	\left[\begin{matrix}
			0 & 0 \\
			7 & 5 \\
			2 & 1 \\
		\end{matrix} \right]
	=
	\left[\begin{matrix}
			1-0 & 3-0 \\
			1-7 & 0-5 \\
			1-2 & 2-1 \\
		\end{matrix} \right]
	=
	\left[\begin{matrix}
			1  & 3  \\
			-6 & -5 \\
			-1 & 1  \\
		\end{matrix} \right]
\end{align}

\begin{lstlisting}[language=Java]
public class Matrix {
    public static void main(String[] args) {
        int[][] A = {
            {1, 3},
            {1, 0},
            {1, 2}
        };
        int[][] B = {
            {0, 0},
            {7, 5},
            {2, 1}
        };
        int[][] C = new int[3][2];

        System.out.println("矩阵加法");
        for(int i = 0; i < 3; i++) {
            for(int j = 0; j < 2; j++) {
                C[i][j] = A[i][j] + B[i][j];
                System.out.print(String.format("%3d", C[i][j]));
            }
            System.out.println();
        }

        System.out.println("矩阵减法");
        for(int i = 0; i < 3; i++) {
            for(int j = 0; j < 2; j++) {
                C[i][j] = A[i][j] - B[i][j];
                System.out.print(String.format("%3d", C[i][j]));
            }
            System.out.println();
        }
    }
}
\end{lstlisting}

\begin{tcolorbox}
	\mybox{运行结果}
	\begin{verbatim}
矩阵加法
1  3
8  5
3  3
矩阵减法
1  3
-6 -5
-1  1
	\end{verbatim}
\end{tcolorbox}

\newpage

\section{字符}

\subsection{字符(Character)}

单个的字符是一种特殊的类型,是用单引号表示字符字面量。每一个字符都有其对应的码值。 \\

ASCII全称American Standard Code for Information Interchange(美国信息交换标准代码),一共定义了128个字符。

\begin{longtable}{|c|c|c|c|c|c|c|c|}
	\hline
	\textbf{ASCII} & \textbf{字符} & \textbf{ASCII} & \textbf{字符} & \textbf{ASCII} & \textbf{字符}          & \textbf{ASCII} & \textbf{字符}          \\
	\hline
	0              & NUT           & 32             & (space)       & 64             & @                      & 96             & \lstinline|`| \\
	\hline
	1              & SOH           & 33             & !             & 65             & A                      & 97             & a                      \\
	\hline
	2              & STX           & 34             & \text{"}      & 66             & B                      & 98             & b                      \\
	\hline
	3              & ETX           & 35             & \#            & 67             & C                      & 99             & c                      \\
	\hline
	4              & EOT           & 36             & \$            & 68             & D                      & 100            & d                      \\
	\hline
	5              & ENQ           & 37             & \%            & 69             & E                      & 101            & e                      \\
	\hline
	6              & ACK           & 38             & \&            & 70             & F                      & 102            & f                      \\
	\hline
	7              & BEL           & 39             & \text{'}      & 71             & G                      & 103            & g                      \\
	\hline
	8              & BS            & 40             & (             & 72             & H                      & 104            & h                      \\
	\hline
	9              & HT            & 41             & )             & 73             & I                      & 105            & i                      \\
	\hline
	10             & LF            & 42             & *             & 74             & J                      & 106            & j                      \\
	\hline
	11             & VT            & 43             & +             & 75             & K                      & 107            & k                      \\
	\hline
	12             & FF            & 44             & ,             & 76             & L                      & 108            & l                      \\
	\hline
	13             & CR            & 45             & -             & 77             & M                      & 109            & m                      \\
	\hline
	14             & SO            & 46             & .             & 78             & N                      & 110            & n                      \\
	\hline
	15             & SI            & 47             & /             & 79             & O                      & 111            & o                      \\
	\hline
	16             & DLE           & 48             & 0             & 80             & P                      & 112            & p                      \\
	\hline
	17             & DC1           & 49             & 1             & 81             & Q                      & 113            & q                      \\
	\hline
	18             & DC2           & 50             & 2             & 82             & R                      & 114            & r                      \\
	\hline
	19             & DC3           & 51             & 3             & 83             & S                      & 115            & s                      \\
	\hline
	20             & DC4           & 52             & 4             & 84             & T                      & 116            & t                      \\
	\hline
	21             & NAK           & 53             & 5             & 85             & U                      & 117            & u                      \\
	\hline
	22             & SYN           & 54             & 6             & 86             & V                      & 118            & v                      \\
	\hline
	23             & TB            & 55             & 7             & 87             & W                      & 119            & w                      \\
	\hline
	24             & CAN           & 56             & 8             & 88             & X                      & 120            & x                      \\
	\hline
	25             & EM            & 57             & 9             & 89             & Y                      & 121            & y                      \\
	\hline
	26             & SUB           & 58             & :             & 90             & Z                      & 122            & z                      \\
	\hline
	27             & ESC           & 59             & ;             & 91             & [                      & 123            & \{                     \\
			\hline
	28             & FS            & 60             & <             & 92             & /                      & 124            & |                      \\
			\hline
	29             & GS            & 61             & =             & 93             & ]                      & 125            & \}                     \\
	\hline
	30             & RS            & 62             & >             & 94             & \lstinline|^| & 126            & \lstinline|~| \\
	\hline
	31             & US            & 63             & ?             & 95             & \_                     & 127            & DEL                    \\
	\hline
	\caption{ASCII码表}
\end{longtable}

\mybox{ASCII码}

\begin{lstlisting}[language=Java]
public class ASCII {
	public static void main(String[] args) {
		for(int i = 0; i < 128; i++) {
			System.out.println(String.format("%c = %d", i, i));
		}
	}
}
\end{lstlisting}

\newpage

\section{字符串}

\subsection{字符串(String)}

字符串是用双引号所表示的0个或多个字符的组合。字符串变量使用String表示,String是一个类,String的变量是对象的管理者而非所有者。

\begin{figure}[H]
	\centering
	\begin{tikzpicture}[scale=1]
		\draw[-] (-5,0) -- (-1,0);
		\draw[-] (1,0) -- (5,0);
		\draw (0,0) node {b = a};

		\draw (-4,-2.5) rectangle (-2,-1.5);
		\draw (2,-2.5) rectangle (4,-1.5);
		\draw (-3,-1) node {int a;};
		\draw (3,-1) node {int b;};
		\draw (-3,-2) node {32};
		\draw (3,-2) node {32};

		\draw (-4,-5.5) rectangle (-2,-4.5);
		\draw (2,-5.5) rectangle (4,-4.5);
		\draw (-3,-4) node {String a;};
		\draw (3,-4) node {String b;};

		\draw (-1,-7) rectangle (1,-6);
		\draw[->] (-2,-5) -- (-1,-6.5);
		\draw[->] (2,-5) -- (1,-6.5);
	\end{tikzpicture}
	\caption{字符串引用}
\end{figure}

通过调用Scanner类中的nextLine()可以获取用户输入的字符串。 \\

\mybox{创建字符串对象}

\begin{lstlisting}[language=Java]
import java.util.Scanner;

public class StringObj {
	public static void main(String[] args) {
		Scanner scanner = new Scanner(System.in);
		System.out.print("输入字符串:");
		String str = scanner.nextLine();
		System.out.println(str);
		scanner.close();
	}
}
\end{lstlisting}

\begin{tcolorbox}
	\mybox{运行结果}
	\begin{verbatim}
输入字符串:Hello World!
Hello World!
	\end{verbatim}
\end{tcolorbox}

\vspace{0.5cm}

\subsection{字符串比较}

字符串的比较分为两种:

\begin{enumerate}
	\item 【==】运算符用于比较是否是同一个对象。
	\item equals()用于比较字符串的内容是否相同。
\end{enumerate}

\mybox{字符串比较}

\begin{lstlisting}[language=Java]
public class StringEqual {
	public static void main(String[] args) {
		String s1 = new String("hello");
		String s2 = new String("hello");

		if(s1 == s2) {
			System.out.println("s1和s2是同一个对象");
		}

		if(s1.equals(s2)) {
			System.out.println("s1与s2内容相同");
		}

		if(s1.equalsIgnoreCase(s2)) {
			System.out.println("s1与s2忽略大小写内容相同");
		}
	}
}
\end{lstlisting}

\begin{tcolorbox}
	\mybox{运行结果}
	\begin{verbatim}
s1与s2内容相同
s1与s2忽略大小写内容相同
	\end{verbatim}
\end{tcolorbox}

\subsection{字符串操作}

字符串是对象,它包含了一系列的常用操作,对它的所有操作都是通过【.】运算符进行的。

\subsubsection{length():计算字符串长度}

\vspace{0.5cm}

\mybox{计算字符串长度}

\begin{lstlisting}[language=Java]
public class StringLength {
	public static void main(String[] args) {
		String str = "Hello World!";
		System.out.println(str.length());
	}
}
\end{lstlisting}

\begin{tcolorbox}
	\mybox{运行结果}
	\begin{verbatim}
12
	\end{verbatim}
\end{tcolorbox}

\subsubsection{charAt():访问字符串中的字符}

字符串中的每一个下标位置都是一个单个的字符,下标的范围从0到length() - 1。

\vspace{0.5cm}

\mybox{访问字符串中的字符}

\begin{lstlisting}[language=Java]
public class CharAt {
	public static void main(String[] args) {
		String str = "Hello World!";
		System.out.println(str.charAt(4));
	}
}
\end{lstlisting}

\begin{tcolorbox}
	\mybox{运行结果}
	\begin{verbatim}
o
	\end{verbatim}
\end{tcolorbox}

\vspace{0.5cm}

\mybox{计算字符串中某个字符出现的次数}

\begin{lstlisting}[language=Java]
import java.util.Scanner;

public class CountOccurence {
	public static void main(String[] args) {
		Scanner scanner = new Scanner(System.in);
		int cnt = 0;		// 出现次数

		System.out.print("输入字符串:");
		String str = scanner.nextLine();
		System.out.print("输入待统计字符:");
		char c = scanner.nextLine().charAt(0);

		int n = str.length();
		for(int i = 0; i < n; i++) {
			if(str.charAt(i) == c) {
				cnt++;
			}
		}

		System.out.println(c + "在" + str + "中出现了" + cnt + "次");
		scanner.close();
	}
}
\end{lstlisting}

\begin{tcolorbox}
	\mybox{运行结果}
	\begin{verbatim}
输入字符串:Hello World
输入待统计字符:l
l在Hello World中出现了3次
	\end{verbatim}
\end{tcolorbox}

\subsubsection{substring():获取子串}

\begin{itemize}
	\item substring(n):获取第n个位置到末尾的全部内容。
	\item substring(begin, end):获取从begin到end位置之前的内容。
\end{itemize}

\mybox{获取子串}

\begin{lstlisting}[language=Java]
public class Substring {
	public static void main(String[] args) {
		String str = "Hello World!";
		System.out.println(str.substring(6));
		System.out.println(str.substring(3, 10));
	}
}
\end{lstlisting}

\begin{tcolorbox}
	\mybox{运行结果}
	\begin{verbatim}
World!
lo Worl
	\end{verbatim}
\end{tcolorbox}

\subsubsection{indexOf():查找}

\begin{itemize}
	\item indexOf(c):获取字符c所在的位置,返回-1表示不存在。
	\item indexOf(c, n):从第n个位置开始查找字符c。
	\item indexOf(t):获取字符串t所在的位置。
\end{itemize}

\mybox{查找}

\begin{lstlisting}[language=Java]
public class IndexOf {
	public static void main(String[] args) {
		String str = "Hello World!";
		System.out.println(str.indexOf('o'));
		System.out.println(str.indexOf('l', 4));
		System.out.println(str.indexOf("llo"));
	}
}
\end{lstlisting}

\begin{tcolorbox}
	\mybox{运行结果}
	\begin{verbatim}
4
9
2
	\end{verbatim}
\end{tcolorbox}

\subsubsection{大小写转换}

\begin{itemize}
	\item toLowerCase():将字符串转换为小写。
	\item toUpperCase():将字符串转换为大写。
\end{itemize}

\mybox{大小写转换}

\begin{lstlisting}[language=Java]
public class UpperLowerCase {
	public static void main(String[] args) {
		String str = "Hello World!";
		System.out.println(str.toLowerCase());
		System.out.println(str.toUpperCase());
	}
}
\end{lstlisting}

\begin{tcolorbox}
	\mybox{运行结果}
	\begin{verbatim}
hello world
HELLO WORLD
	\end{verbatim}
\end{tcolorbox}

\subsubsection{替换}

\begin{itemize}
	\item replace(c1, c2):将所有字符c1替换为字符c2,返回新字符串。
	\item replace(s1, s2):将所有子串s1替换为子串s2,返回新字符串。
\end{itemize}

\mybox{字符串替换}

\begin{lstlisting}[language=Java]
public class Replace {
	public static void main(String[] args) {
		String str = "Hello World!";
		System.out.println(str.replace('l', '*'));
		System.out.println(str.replace("ll", "##"));
	}
}
\end{lstlisting}

\begin{tcolorbox}
	\mybox{运行结果}
	\begin{verbatim}
He**o Wor*d!
He##o World!
	\end{verbatim}
\end{tcolorbox}

\subsubsection{split(regex):字符串分割}

根据匹配给定的正则表达式拆分字符串,返回拆分后的字符串数组。 \\

\mybox{字符串替换}

\begin{lstlisting}[language=Java]
public class Split {
	public static void main(String[] args) {
		String str = "This is a string.";
		String[] s = str.split(" ");
		for(String item : s) {
			System.out.println(item);
		}
	}
}
\end{lstlisting}

\begin{tcolorbox}
	\mybox{运行结果}
	\begin{verbatim}
This
is
a
string.
	\end{verbatim}
\end{tcolorbox}

\vspace{0.5cm}

\mybox{统计单词个数}

\begin{lstlisting}[language=Java]
import java.util.Scanner;

public class CountWord {
	public static void main(String[] args) {
		Scanner scanner = new Scanner(System.in);
		System.out.print("输入英语句子:");
		String str = scanner.nextLine();
		// "\\s+"表示一个或多个空格、回车、制表符等空白符
		String[] words = str.split("\\s+");
		System.out.println("单词个数:" + words.length);
		for(String word : words) {
			System.out.println("\t" + word);
		}
		scanner.close();
	}
}
\end{lstlisting}

\begin{tcolorbox}
	\mybox{运行结果}
	\begin{verbatim}
输入英语句子:This is a string.
单词个数:4
	This
	is
	a
	string.
	\end{verbatim}
\end{tcolorbox}

\newpage