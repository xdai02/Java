\chapter{数组}

\section{一维数组}

\subsection{数组(Array)}

一个变量只能存储一个内容,如果需要存储更多数据,就需要使用数组解决问题。一个数组变量可以存放多个数据,数组是一个值的集合,它们共享同一个名字,数组中的每个变量都能被其下标所访问。

\vspace{-0.5cm}

\begin{lstlisting}[language=Java]
int[] number = new int[10];
float[] grade = new float[50];
\end{lstlisting}

\begin{figure}[H]
	\centering
	\begin{tikzpicture}[scale=0.5]
		\draw[-] (0,0) -- (5,0) -- (10,0) -- (15,0) -- (20,0) -- (25,0) -- (25,3) -- (20,3) -- (15,3) -- (10,3) -- (5,3) -- (0,3) -- (0,0);
		\draw[-] (5,0) -- (5,3);
		\draw[-] (10,0) -- (10,3);
		\draw[-] (15,0) -- (15,3);
		\draw[-] (20,0) -- (20,3);

		\draw (2.5,1.5) node {a[0]};
		\draw (7.5,1.5) node {a[1]};
		\draw (12.5,1.5) node {a[2]};
		\draw (17.5,1.5) node {a[3]};
		\draw (22.5,1.5) node {a[4]};
	\end{tikzpicture}
\end{figure}

\begin{itemize}
	\item 元素:数组中的每个变量
	\item 大小:数组的容量
	\item 下标 / 索引(index):元素的位置,下标从0开始,必须为非负整数
\end{itemize}

\subsection{数组初始化}

一维数组可以在声明时进行初始化:

\vspace{-0.5cm}

\begin{lstlisting}[language=Java]
int[] arr = {3, 6, 8, 2, 4, 0, 9, 7, 1, 5};
int[] arr = new int[] {3, 6, 8, 2, 4, 0, 9, 7, 1, 5};
\end{lstlisting}

很多时候在使用数组之前需要将数组的内容全部清空,这可以利用循环来实现。 \\

\mybox{一维数组初始化}

\begin{lstlisting}[language=Java]
public class InitArr {
    public static void main(String[] args) {
        int[] arr = new int[100];
        for(int i = 0; i < arr.length; i++) {
            arr[i] = 0;
        }
    }
}
\end{lstlisting}

\vspace{0.5cm}

\mybox{数组最大值和最小值}

\begin{lstlisting}[language=Java]
public class MaxMin {
    public static void main(String[] args) {
        int[] num = {7, 6, 2, 9, 3, 1, 4, 0, 5, 8};
        int max = num[0];
        int min = num[0];

        for(int i = 1; i < num.length; i++) {
            if(num[i] > max) {
                max = num[i];
            } else if(num[i] < min) {
                min = num[i];
            }
        }

        System.out.println("max = " + max);
        System.out.println("min = " + min);
    }
}
\end{lstlisting}

\begin{tcolorbox}
	\mybox{运行结果} \\
	max = 9
	min = 0
\end{tcolorbox}

\subsection{for-each}

for-each环是for循环的特殊简化版。

\begin{lstlisting}[language=Java]
for(dataType var : set) {
    // code
}
\end{lstlisting}

\mybox{遍历数组}

\begin{lstlisting}[language=Java]
public class ForEach {
    public static void main(String[] args) {
        int[] arr = {7, 6, 2, 9, 3, 1, 4, 0, 5, 8};
        for(int elem : arr) {
            System.out.print(elem + " ");
        }
    }
}
\end{lstlisting}

\begin{tcolorbox}
	\mybox{运行结果} \\
	7 6 2 9 3 1 4 0 5 8
\end{tcolorbox}

\newpage

\section{二维数组}

\subsection{二维数组(2D Array)}

二维数组包括行和列两个维度,可以看成是由多个一维数组组成。

\begin{table}[H]
	\centering
	\setlength{\tabcolsep}{5mm}{
		\begin{tabular}{|c|c|c|c|}
			\hline
			a[0][0] & a[0][1] & a[0][2] & a[0][3] \\
			\hline
			a[1][0] & a[1][1] & a[1][2] & a[1][3] \\
			\hline
			a[2][0] & a[2][1] & a[2][2] & a[2][3] \\
			\hline
		\end{tabular}
	}
\end{table}

二维数组可以在声明时进行初始化:

\begin{lstlisting}[language=Java]
int[][] arr = new int[2][3];
int[][] arr = {{1, 2, 3}, {4, 5, 6}};
\end{lstlisting}

\vspace{0.5cm}

\mybox{初始化二维数组}

\begin{lstlisting}[language=Java]
public class Init2dArr {
    public static void main(String[] args) {
        int[][] arr = new int[3][4];
        for(int i = 0; i < arr.length; i++) {
            for(int j = 0; j < arr[i].length; j++) {
                arr[i][j] = 0;
            }
        }
    }
}
\end{lstlisting}

\vspace{0.5cm}

\mybox{矩阵运算}

矩阵的加法/减法是指两个矩阵把其相对应元素进行加减的运算。 \\

矩阵加法:两个$ m \times n $矩阵A和B的和,标记为$ A + B $,结果为一个$ m \times n $的矩阵,其内的各元素为其相对应元素相加后的值。 \\

矩阵减法:两个$ m \times n $矩阵A和B的差,标记为$ A - B $,结果为一个$ m \times n $的矩阵,其内的各元素为其相对应元素相减后的值。

\begin{align}\nonumber
	\left[\begin{matrix}
			1 & 3 \\
			1 & 0 \\
			1 & 2 \\
		\end{matrix} \right]
	+
	\left[\begin{matrix}
			0 & 0 \\
			7 & 5 \\
			2 & 1 \\
		\end{matrix} \right]
	=
	\left[\begin{matrix}
			1+0 & 3+0 \\
			1+7 & 0+5 \\
			1+2 & 2+1 \\
		\end{matrix} \right]
	=
	\left[\begin{matrix}
			1 & 3 \\
			8 & 5 \\
			3 & 3 \\
		\end{matrix} \right]
\end{align}

\begin{align}\nonumber
	\left[\begin{matrix}
			1 & 3 \\
			1 & 0 \\
			1 & 2 \\
		\end{matrix} \right]
	-
	\left[\begin{matrix}
			0 & 0 \\
			7 & 5 \\
			2 & 1 \\
		\end{matrix} \right]
	=
	\left[\begin{matrix}
			1-0 & 3-0 \\
			1-7 & 0-5 \\
			1-2 & 2-1 \\
		\end{matrix} \right]
	=
	\left[\begin{matrix}
			1  & 3  \\
			-6 & -5 \\
			-1 & 1  \\
		\end{matrix} \right]
\end{align}

\begin{lstlisting}[language=Java]
public class Matrix {
    public static void main(String[] args) {
        int[][] A = {
            {1, 3},
            {1, 0},
            {1, 2}
        };
        int[][] B = {
            {0, 0},
            {7, 5},
            {2, 1}
        };
        int[][] C = new int[3][2];

        System.out.println("矩阵加法");
        for(int i = 0; i < 3; i++) {
            for(int j = 0; j < 2; j++) {
                C[i][j] = A[i][j] + B[i][j];
                System.out.print(String.format("%3d", C[i][j]));
            }
            System.out.println();
        }

        System.out.println("矩阵减法");
        for(int i = 0; i < 3; i++) {
            for(int j = 0; j < 2; j++) {
                C[i][j] = A[i][j] - B[i][j];
                System.out.print(String.format("%3d", C[i][j]));
            }
            System.out.println();
        }
    }
}
\end{lstlisting}

\begin{tcolorbox}
	\mybox{运行结果} \\
	矩阵加法 \\
    1  3 \\
    8  5 \\
    3  3 \\
    矩阵减法 \\
    1  3 \\
    -6 -5 \\
    -1  1
\end{tcolorbox}

\newpage