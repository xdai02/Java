\chapter{函数}

\section{函数}

\subsection{函数(Function)}

函数执行一个特定的任务,JS提供了大量内置函数,例如alert()用来显示警告对话框、parseInt()用来将字符串转换为整型等。

\begin{figure}[H]
	\centering
	\begin{tikzpicture}[scale=0.5]
		\draw[-] (5,-2) -- (10,-2) -- (10,2) -- (5,2) -- (5,-2);
		\draw[->] (0,0) -- (5,0);
		\draw[->] (10,0) -- (15,0);

		\draw (-2,0) node {Input};
		\draw (17,0) node {Output};
		\draw (7.5,0) node {Function};
	\end{tikzpicture}
	\caption{函数}
\end{figure}

当调用函数时,程序控制权会转移给被调用的函数,当函数执行结束后,函数会把程序序控制权交还给其调用者。

\begin{figure}[H]
	\centering
	\begin{tikzpicture}[]
		\draw (0,4.5) node {Caller};
		\draw[->] (0,4) -- (0,0.5);
		\draw[->] (0,-0.5) -- (0,-4);
		\draw (0,0) node {调用foo()};

		\draw (4,4) node {foo()};
		\draw[->] (4,3) -- (4,0.5);
		\draw[->] (4,-0.5) -- (4,-3);
		\draw (4,0) node {调用bar()};

		\draw (8,3) node {bar()};
		\draw[->] (8,2) -- (8,-2);

		\draw[->] (0.5,0.5) -- (3.5,3);
		\draw[->] (3.5,-3) -- (0.5,-0.5);
		\draw[->] (4.5,0.5) -- (7.5,2);
		\draw[->] (7.5,-2) -- (4.5,-0.5);
	\end{tikzpicture}
	\caption{函数调用}
\end{figure}

函数的定义需要使用关键字function,函数的参数列表包括参数的类型、顺序、数量等信息,参数列表可以为空。 \\

\begin{lstlisting}[style=htmlcssjs]
function funcName(parameterList) {
    // code
}
\end{lstlisting}

\subsection{函数设计方法}

为什么不把所有的代码全部写在一起,还需要自定义函数呢? \\

使用函数有以下好处:

\begin{enumerate}
	\item 避免代码复制,代码复制是程序质量不良的表现
	\item 便于代码维护
	\item 避免重复造轮子,提高开发效率
\end{enumerate}

在设计函数的时候需要考虑以下的几点要素:

\begin{enumerate}
	\item 确定函数的功能

	\item 确定函数的参数
	      \begin{itemize}
		      \item 是否需要参数
		      \item 参数个数
		      \item 参数类型
	      \end{itemize}

	\item 确定函数的返回值
	      \begin{itemize}
		      \item 是否需要返回值
		      \item 返回值类型
	      \end{itemize}
\end{enumerate}

\mybox{函数实现返回最大值} \\

\begin{lstlisting}[style=htmlcssjs]
function max(num1, num2) {
    // if(num1 > num2) {
    //  return num1;
    // } else {
    //  return num2;
    // }

    return num1 > num2 ? num1 : num2;
}

console.log(max(4, 12));
console.log(max(54, 33));
console.log(max(0, -12));
console.log(max(-999, -774));
\end{lstlisting}

\begin{tcolorbox}
	\mybox{运行结果} \\
	12 \\
	54 \\
	0 \\
	-774
\end{tcolorbox}

\vspace{0.5cm}
\mybox{函数实现累加和} \\

\begin{lstlisting}[style=htmlcssjs]
function sum(start, end) {
    var total = 0;
    for(var i = start; i <= end; i++) {
        total += i;
    }
    return total;
}

console.log("1-100的累加和 = " + sum(1, 100));
console.log("1024-2048的累加和 = " + sum(1024, 2048));
\end{lstlisting}

\begin{tcolorbox}
	\mybox{运行结果} \\
	1-100的累加和 = 5050 \\
	1024-2048的累加和 = 1574400
\end{tcolorbox}

\vspace{0.5cm}
\mybox{函数实现输出i行j列由自定义字符组成的图案} \\

\begin{lstlisting}[style=htmlcssjs]
function print_chars(row, col, c) {
    var str = "";
    for(var i = 0; i < row; i++) {
        for(var j = 0; j < col; j++) {
            str += c;
        }
        str += "\n";
    }
    console.log(str);
}

print_chars(5, 10, '?');
\end{lstlisting}

\begin{tcolorbox}
	\mybox{运行结果} \\
	?????????? \\
	?????????? \\
	?????????? \\
	?????????? \\
	??????????
\end{tcolorbox}

\newpage

\section{局部变量与全局变量}

\subsection{局部变量(Local Varaible)}

JS的局部变量是在函数里面被声明的,这些变量的作用域在本地,也就是说这些变量只能在函数内部可用。本地变量在函数调用时被创造,在函数结束时被销毁。 \\

在函数中,函数的每次调用就会产生一个独立的空间,在这个空间中的变量,是函数的这次运行所独有的,函数的参数也是局部变量。 \\

\mybox{局部变量} \\

\begin{lstlisting}[style=htmlcssjs]
function test(a) {
    a = 2;
    console.log("a = " + a);
}

var a = 1;
console.log("a = " + a);
test(a);
console.log("a = " + a);
\end{lstlisting}

\begin{tcolorbox}
	\mybox{运行结果} \\
	a = 1 \\
	a = 2 \\
	a = 1
\end{tcolorbox}

\subsection{全局变量(Global Varaible)}

JS的全局变量就是在函数外被声明的变量,作用域为全局,所有的在页面上的脚本和函数都可以获取这些变量。全局变量在其被声明时创建,在页面被关闭时被销毁。 \\

全局变量的优先级低于局部变量,当全局变量与局部变量重名的时候,起作用的是局部变量,全局变量会被暂时屏蔽掉。

\begin{figure}[H]
	\centering
	\begin{tikzpicture}[]
		\draw (4,4.5) node {全局};
		\draw (0,0) rectangle (8,4);

		\draw (2.5,3.5) node {局部};
		\draw (1,1) rectangle (4,3);

		\draw (2.5,2.5) node {变量A};
		\draw (2,1.5) rectangle (3,2);

		\draw (6,2.5) node {变量B};
		\draw (5.5,1.5) rectangle (6.5,2);
	\end{tikzpicture}
	\caption{全局变量}
\end{figure}

\mybox{全局变量} \\

\begin{lstlisting}[style=htmlcssjs]
var a = 1;          // 全局变量

function test() {
    var a = 2;      // 本地变量
    console.log("a = " + a);
}

test();
\end{lstlisting}

\begin{tcolorbox}
	\mybox{运行结果} \\
	a = 2
\end{tcolorbox}

\newpage

\section{递归} \label{recursive}

\subsection{递归(Recursion)}

要理解递归,先得理解递归(见\ref{recursive}章节)。 \\

在函数的内部,直接或者间接的调用自己的过程就叫作递归。对于一些问题,使用递归可以简洁易懂的解决问题,但是递归的缺点是性能低,占用大量系统栈空间。 \\

递归算法很多时候可以处理一些特别复杂、难以直接解决的问题。例如:

\begin{itemize}
	\item 迷宫
	\item 汉诺塔
	\item 八皇后
	\item 排序
	\item 搜索
\end{itemize}

在定义递归函数时,一定要确定一个结束条件,否则会造成无限递归的情况,最终会导致栈溢出。

\begin{figure}[H]
	\centering
	\includegraphics[scale=0.7]{img/C13/13-3/1.png}
\end{figure}

\begin{figure}[H]
	\centering
	\includegraphics[scale=0.6]{img/C13/13-3/2.png}
\end{figure}

\begin{figure}[H]
	\centering
	\includegraphics[scale=0.6]{img/C13/13-3/3.png}
\end{figure}

\begin{figure}[H]
	\centering
	\includegraphics[scale=1.3]{img/C13/13-3/4.png}
\end{figure}

\begin{figure}[H]
	\centering
	\includegraphics[scale=0.6]{img/C13/13-3/5.png}
\end{figure}

\mybox{无限递归} \\

\begin{lstlisting}[style=htmlcssjs]
function tell_story() {
    console.log("从前有座山");
    console.log("山里有座庙");
    console.log("庙里有个老和尚和小和尚");
    console.log("老和尚对小和尚在讲故事");
    console.log("他讲的故事是:");
    tell_story();
}

tell_story();
\end{lstlisting}

\begin{tcolorbox}
	\mybox{运行结果} \\
	从前有座山 \\
	山里有座庙 \\
	庙里有个老和尚和小和尚 \\
	老和尚对小和尚在讲故事 \\
	他讲的故事是: \\
	从前有座山 \\
	山里有座庙 \\
	庙里有个老和尚和小和尚 \\
	老和尚对小和尚在讲故事 \\
	他讲的故事是: \\
	...
\end{tcolorbox}

递归函数一般需要定义递归的出口,即结束条件,确保递归能够在适合的地方退出。 \\

\mybox{阶乘} \\

\begin{lstlisting}[style=htmlcssjs]
function factorial(n) {
    if(n == 0 || n == 1) {
        return 1;
    }
    return n * factorial(n - 1);
}

console.log("5! = " + factorial(5));
\end{lstlisting}

\begin{tcolorbox}
	\mybox{运行结果} \\
	5! = 120
\end{tcolorbox}

\begin{figure}[H]
	\centering
	\begin{tikzpicture}[]
		\draw (0,0) rectangle (3,1.5);
		\draw (3,-2) rectangle (6,-0.5);
		\draw (6,-4) rectangle (9,-2.5);
		\draw (9,-6) rectangle (12,-4.5);
		\draw (12,-8) rectangle (15,-6.5);

		\draw (12.75,-10.75) rectangle (14.25,-9.25);
		\draw (9.75,-8.75) rectangle (11.25,-7.25);
		\draw (6.75,-6.75) rectangle (8.25,-5.25);
		\draw (3.75,-4.75) rectangle (5.25,-3.25);
		\draw (0.75,-2.75) rectangle (2.25,-1.25);

		\draw (1.5,0.75) node {$ factorial(5) $};
		\draw (4.5,-1.25) node {$ factorial(4) $};
		\draw (7.5,-3.25) node {$ factorial(3) $};
		\draw (10.5,-5.25) node {$ factorial(2) $};
		\draw (13.5,-7.25) node {$ factorial(1) $};

		\draw (13.5,-10) node {$ 1 $};
		\draw (10.5,-8) node {$ 2 $};
		\draw (7.5,-6) node {$ 6 $};
		\draw (4.5,-4) node {$ 24 $};
		\draw (1.5,-2) node {$ 120 $};

		\draw[->] (3,0.75) -- (4.5,0.75) -- (4.5,-0.5);
		\draw[->] (6,-1.25) -- (7.5,-1.25) -- (7.5,-2.5);
		\draw[->] (9,-3.25) -- (10.5,-3.25) -- (10.5,-4.5);
		\draw[->] (12,-5.25) -- (13.5,-5.25) -- (13.5,-6.5);

		\draw[->] (12.75,-10) -- (10.5,-10) -- (10.5,-8.75);
		\draw[->] (9.75,-8) -- (7.5,-8) -- (7.5,-6.75);
		\draw[->] (6.75,-6) -- (4.5,-6) -- (4.5,-4.75);
		\draw[->] (3.75,-4) -- (1.5,-4) -- (1.5,-2.75);

		\draw (4.5,1) node {$ 5 * factorial(4) $};
		\draw (7.5,-1) node {$ 4 * factorial(3) $};
		\draw (10.5,-3) node {$ 3 * factorial(2) $};
		\draw (13.5,-5) node {$ 2 * factorial(1) $};

		\draw (11,-10.5) node {$ 2 * 1 $};
		\draw (8,-8.5) node {$ 3 * 2 $};
		\draw (5,-6.5) node {$ 4 * 6 $};
		\draw (2,-4.5) node {$ 5 * 24 $};
	\end{tikzpicture}
	\caption{阶乘}
\end{figure}

\mybox{斐波那契数列(递归)} \\

\begin{lstlisting}[style=htmlcssjs]
function fibonacci(n) {
    if(n == 1 || n == 2) {
        return 1;
    }
    return fibonacci(n - 2) + fibonacci(n - 1);
}

n = 7;
console.log("斐波那契数列第" + n + "位:" + fibonacci(7));
\end{lstlisting}

\begin{tcolorbox}
	\mybox{运行结果} \\
	斐波那契数列第7位:13
\end{tcolorbox}

\begin{figure}[H]
	\centering
	\begin{tikzpicture}[
			level distance=2.5cm,
			level 1/.style={sibling distance=6cm},
			level 2/.style={sibling distance=3cm},
			level 3/.style={sibling distance=2cm}
		]
		\node {$ f(5) $}
		child {
				node {$ f(3) $}
				child {node {$ f(1) $}}
				child {
						node {$ f(2) $}
						child {node {$ f(0) $}}
						child {node {$ f(1) $}}
					}
			}
		child {
				node {$ f(4) $}
				child {
						node {$ f(2) $}
						child {node {$ f(0) $}}
						child {node {$ f(1) $}}
					}
				child {
						node {$ f(3) $}
						child {node {$ f(1) $}}
						child {
								node {$ f(2) $}
								child {node {$ f(0) $}}
								child {node {$ f(1) $}}
							}
					}
			};
	\end{tikzpicture}
	\caption{递归树}
\end{figure}

\mybox{斐波那契数列(迭代)} \\

\begin{lstlisting}[style=htmlcssjs]
function fibonacci(n) {
    var f = new Array(n + 1);
    f[1] = f[2] = 1;
    for(var i = 3; i <= n; i++) {
        f[i] = f[i - 2] + f[i - 1];
    }
    return f[n];
}

n = 7;
console.log("斐波那契数列第" + n + "位:" + fibonacci(7));
\end{lstlisting}

\begin{tcolorbox}
	\mybox{运行结果} \\
	斐波那契数列第7位:13
\end{tcolorbox}

\mybox{阿克曼函数}

\begin{align}\nonumber
	A(m, n) =
	\begin{cases}
		n + 1             & m = 0        \\
		A(m-1, 1)         & m > 0, n = 0 \\
		A(m-1, A(m, n-1)) & m > 0, n > 0 \\
	\end{cases}
\end{align}

\begin{lstlisting}[style=htmlcssjs]
function A(m, n) {
    if(m == 0) {
        return n + 1;
    } else if(m > 0 && n == 0) {
        return A(m-1, 1);
    } else if(m > 0 && n > 0) {
        return A(m-1, A(m, n-1));
    }
}

console.log(A(3, 4));
\end{lstlisting}

\begin{tcolorbox}
	\mybox{运行结果} \\
	125
\end{tcolorbox}

\begin{table}[H]
	\centering
	\setlength{\tabcolsep}{0.5mm}{
		\begin{tabular}{|c|c|c|c|c|c|c|}
			\hline
			\diagbox{$ m $}{$ n $} & \textbf{$ 0 $} & \textbf{$ 1 $}    & \textbf{$ 2 $}    & \textbf{$ 3 $}          & \textbf{$ 4 $}    & \textbf{$ n $}                                         \\
			\hline
			\textbf{$ 0 $}         & $ 1 $          & $ 2 $             & $ 3 $             & $ 4 $                   & $ 5 $             & $ n + 1 $                                              \\
			\hline
			\textbf{$ 1 $}         & $ 2 $          & $ 3 $             & $ 4 $             & $ 5 $                   & $ 6 $             & $ 2 + (n + 3) - 3 $                                    \\
			\hline
			\textbf{$ 2 $}         & $ 3 $          & $ 5 $             & $ 7 $             & $ 9 $                   & $ 11 $            & $ 2(n + 3) - 3 $                                       \\
			\hline
			\textbf{$ 3 $}         & $ 5 $          & $ 13 $            & $ 29 $            & $ 61 $                  & $ 125 $           & $ 2^{n + 3} - 3 $                                      \\
			\hline
			\textbf{$ 4 $}         & $ 13 $         & $ 65533 $         & $ 2^{65536} - 3 $ & $ A(3, 2^{65536} - 3) $ & $ A(3, A(4, 3)) $ & $ \underbrace{2^{2^{.^{.^{.{^2}}}}}}_{n+3\ twos} - 3 $ \\
			\hline
			\textbf{$ 5 $}         & $ 65533 $      & $ A(4, 65533) $   & $ A(4, A(5, 1)) $ & $ A(4, A(5, 2)) $       & $ A(4, A(5, 3)) $ & $ \dots $                                              \\
			\hline
			\textbf{$ 6 $}         & $ A(5, 1) $    & $ A(5, A(5, 1)) $ & $ A(5, A(6, 1)) $ & $ A(5, A(6, 2)) $       & $ A(5, A(6, 3)) $ & $ \dots $                                              \\
			\hline
		\end{tabular}
	}
	\caption{阿克曼函数}
\end{table}

\begin{figure}[H]
	\centering
	\includegraphics[]{img/C13/13-3/6.png}
\end{figure}

\mybox{汉诺塔} \\

给定三根柱子,其中A柱子从大到小套有n个圆盘,问题是如何借助B柱子,将圆盘从A搬到C。 \\

规则:

\begin{itemize}
	\item 一次只能搬动一个圆盘
	\item 不能将大圆盘放在小圆盘上面
\end{itemize}

\begin{figure}[H]
	\centering
	\includegraphics[scale=0.4]{img/C13/13-3/7.png}
\end{figure}

递归算法求解汉诺塔问题:

\begin{enumerate}
	\item 将前n-1个圆盘从A柱借助于C柱搬到B柱。
	\item 将最后一个圆盘直接从A柱搬到C柱。
	\item 将n-1个圆盘从B柱借助于A柱搬到C柱。
\end{enumerate}

\begin{lstlisting}[style=htmlcssjs]
var move = 0;       // 移动次数

/**
 * @brief   汉诺塔算法
 * @note    把 n 个盘子从 src 借助 mid 移到 dst
 * @param   n: 层数
 * @param   src: 起点柱子
 * @param   mid: 临时柱子
 * @param   dst: 目标柱子
 */
function hanoi(n, src, mid, dst) {
    if(n == 1) {
        console.log(n + "号盘:" + src + " -> " + dst);
        move++;
    } else {
        // 把前 n-1 个盘子从 src 借助 dst 移到 mid
        hanoi(n-1, src, dst, mid);
        // 移动第 n 个盘子
        console.log(n + "号盘:" + src + " -> " + dst);
        move++;
        // 把刚才的 n-1 个盘子从 mid 借助 src 移到 dst
        hanoi(n-1, mid, src, dst);
    }
}

hanoi(4, 'A', 'B', 'C');
console.log("步数 ==> " + move);
\end{lstlisting}

\begin{tcolorbox}
	\mybox{运行结果} \\
	1号盘:A -> B \\
	2号盘:A -> C \\
	1号盘:B -> C \\
	3号盘:A -> B \\
	1号盘:C -> A \\
	2号盘:C -> B \\
	1号盘:A -> B \\
	4号盘:A -> C \\
	1号盘:B -> C \\
	2号盘:B -> A \\
	1号盘:C -> A \\
	3号盘:B -> C \\
	1号盘:A -> B \\
	2号盘:A -> C \\
	1号盘:B -> C \\
	步数 ==> 15
\end{tcolorbox}

\newpage