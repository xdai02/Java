\chapter{面向对象}

\section{封装}

\subsection{类与对象}

在面向对象编程中,把构成问题的事物分解成各个对象,每个对象都有自己的数据和行为,程序通过对象之间的交互来实现功能。\\

类(class)是一个模板,定义了对象的属性和方法,用来描述同一类对象的共同特征和行为。对象(object)是类的实例,它具有类定义的属性和方法。\\

关键字new可以实例化一个类对象,之后就可以通过访问对象的属性和方法来操作对象。\\

\mybox{银行账户}

\begin{lstlisting}[language=Java]
public class BankAccount {
    String owner;
    String account;
    double balance;

    public void deposit(double amount) {
        balance += amount;
    }

    public void withdraw(double amount) {
        balance -= amount;
    }
}
\end{lstlisting}

\begin{lstlisting}[language=Java]
public class Bank {
    public static void main(String[] args) {
        BankAccount account = new BankAccount();
        account.owner = "Terry";
        account.account = "6250941006528599";
        account.balance = 50;

        System.out.println("Owner: " + account.owner);
        System.out.println("Account: " + account.account);
        System.out.println("Balance: " + account.balance);

        account.deposit(100);
        System.out.println("Balance: " + account.balance);

        account.withdraw(70);
        System.out.println("Balance: " + account.balance);
    }
}
\end{lstlisting}

\begin{tcolorbox}
    \mybox{运行结果}
    \begin{verbatim}
Owner: Terry
Account: 6250941006528599
Balance: 50.0
Balance: 150.0
Balance: 80.0
\end{verbatim}
\end{tcolorbox}

\vspace{0.5cm}

\subsection{封装(Encapsulation)}

封装是面向对象的重要原则,尽可能隐藏对象的内部实现细节。封装可以认为是一个保护屏障,防止该类的数据被外部随意访问。当要访问该类的数据时,必须通过指定的接口。合适的封装可以让代码更容易理解和维护,也加强了程序的安全性。\\

为了实现封装,需要对类的属性和方法进行访问权限的控制:

\begin{enumerate}
    \item public:允许任何地方访问。
    \item private:只允许在类的内部访问。
    \item protected:只允许在类的内部和子类中访问。
    \item default:只允许在同一个包中访问。
\end{enumerate}

通常会将类的属性设置为private,然后对外提供一对setter/getter方法来访问该属性。\\

为了避免方法的参数与类的属性重名造成歧义,可以使用this关键字用来指代当前对象。\\

\mybox{银行账户}

\begin{lstlisting}[language=Java]
public class BankAccount {
    private final int ACCOUNT_DIGITS = 16;

    private String owner;
    private String account;
    private double balance;

    public void setOwner(String owner) {
        if (!owner.isEmpty()) {
            this.owner = owner;
        }
    }

    public String getOwner() {
        return owner;
    }

    public void setaccount(String account) {
        if (account.length() == ACCOUNT_DIGITS) {
            this.account = account;
        }
    }

    public String getAccount() {
        return account;
    }

    public void setBalance(double balance) {
        if (balance >= 0) {
            this.balance = balance;
        }
    }

    public double getBalance() {
        return balance;
    }

    public boolean deposit(double amount) {
        if (amount <= 0) {
            return false;
        }
        balance += amount;
        return true;
    }

    public boolean withdraw(double amount) {
        if (amount <= 0 || amount > balance) {
            return false;
        }
        balance -= amount;
        return true;
    }
}
\end{lstlisting}

\begin{lstlisting}[language=Java]
public class Bank {
    public static void main(String[] args) {
        BankAccount account = new BankAccount();
        account.setOwner("Terry");
        account.setAccount("6250941006528599");
        account.setBalance(50);

        System.out.println("Owner: " + account.getOwner());
        System.out.println("Account: " + account.getAccount());
        System.out.println("Balance: " + account.getBalance());

        account.deposit(100);
        System.out.println("Balance: " + account.getBalance());

        account.withdraw(70);
        System.out.println("Balance: " + account.getBalance());
    }
}
\end{lstlisting}

\begin{tcolorbox}
    \mybox{运行结果}
    \begin{verbatim}
Owner: Terry
Account: 6250941006528599
Balance: 50.0
Balance: 150.0
Balance: 80.0
\end{verbatim}
\end{tcolorbox}

\newpage

\section{构造方法}

\subsection{构造方法(Constructor)}

构造方法是一种特殊的方法,会在创建对象时自动调用,用于创建并初始化对象。每个类可以有一个或多个构造方法,构造方法的名字必须和类名一致。构造方法没有返回值,返回值类型部分不写。

\vspace{-0.5cm}

\begin{lstlisting}[language=Java]
public BankAccount() {
    owner = "admin";
    account = "0000000000000000";
    balance = 0;
}
\end{lstlisting}

如果一个类中没有写构造方法,系统会自动提供一个public的无参构造方法,以便实例化对象。如果一个类中已经写了构造方法,系统将不会再提供默认的无参构造方法。

\vspace{-0.5cm}

\begin{lstlisting}[language=Java]
public BankAccount(String owner, String account, double balance) {
    if (!owner.isEmpty()) {
        this.owner = owner;
    }

    if (account.length() == ACCOUNT_DIGITS) {
        this.account = account;
    }

    if (balance >= 0) {
        this.balance = balance;
    }
}
\end{lstlisting}

\vspace{0.5cm}

\subsection{重载(Overload)}

重载用于在同一个类定义多个同名方法,但是这些方法的参数列表不同。重载的主要用途是提供方法的多种版本,以便满足不同的需求。\\

重载还可以使代码更具可读性,因为它使得方法名更具描述性,而不必考虑特定的参数列表。\\

\mybox{银行账户}

\begin{lstlisting}[language=Java]
public class BankAccount {
    private final int ACCOUNT_DIGITS = 16;

    private String owner;
    private String account;
    private double balance;

    public BankAccount() {
        owner = "admin";
        account = "0000000000000000";
        balance = 0;
    }

    public BankAccount(String owner, String account, double balance) {
        if (!owner.isEmpty()) {
            this.owner = owner;
        }

        if (account.length() == ACCOUNT_DIGITS) {
            this.account = account;
        }

        if (balance >= 0) {
            this.balance = balance;
        }
    }

    public void setOwner(String owner) {
        if (!owner.isEmpty()) {
            this.owner = owner;
        }
    }

    public String getOwner() {
        return owner;
    }

    public void setAccount(String account) {
        if (account.length() == ACCOUNT_DIGITS) {
            this.account = account;
        }
    }

    public String getAccount() {
        return account;
    }

    public void setBalance(double balance) {
        if (balance >= 0) {
            this.balance = balance;
        }
    }

    public double getBalance() {
        return balance;
    }

    public boolean deposit(double amount) {
        if (amount <= 0) {
            return false;
        }
        balance += amount;
        return true;
    }

    public boolean withdraw(double amount) {
        if (amount <= 0 || amount > balance) {
            return false;
        }
        balance -= amount;
        return true;
    }

    public boolean withdraw(double amount, double fee) {
        if(amount <= 0 || amount + fee > balance) {
            return false;
        }

        balance -= amount + fee;
        return true;
    }
}
\end{lstlisting}

\begin{lstlisting}[language=Java]
public class Bank {
    public static void main(String[] args) {
        BankAccount account1 = new BankAccount();
        System.out.println(
                "Account 1 Owner: " + account1.getOwner()
        );
        System.out.println(
                "Account 1 Account: " + account1.getAccount()
        );
        System.out.println(
                "Account 1 Balance: " + account1.getBalance()
        );

        BankAccount account2 = new BankAccount(
                "Terry", "6250941006528599", 50
        );
        System.out.println(
                "Account 2 Balance: " + account2.getBalance()
        );

        account2.withdraw(20);
        System.out.println(
                "Account 2 Balance: " + account2.getBalance()
        );

        account2.withdraw(10, 1);
        System.out.println(
                "Account 2 Balance: " + account2.getBalance()
        );
    }
}
\end{lstlisting}

\begin{tcolorbox}
    \mybox{运行结果}
    \begin{verbatim}
Account 1 Owner: admin
Account 1 Account: 0000000000000000
Account 1 Balance: 0.0
Account 2 Balance: 50.0
Account 2 Balance: 30.0
Account 2 Balance: 19.0  
	\end{verbatim}
\end{tcolorbox}

\newpage

\section{继承}

\subsection{继承(Inheritance)}

继承指一个类可以继承另一个类的特征和行为,并可以对其进行扩展。这样就可以避免在多个类中重复定义相同的特征和行为。\\

\begin{figure}[H]
    \centering
    \begin{tikzpicture}
        \begin{class}[text width = 5cm]{Product}{0,0}
            \attribute{- name}
            \attribute{- price}
            \operation{+ Product(String, double)}
            \operation{+ getName() : String}
            \operation{+ getPrice() : double}
            \operation{+ setName(String) : void}
            \operation{+ setPrice(double) : void}
        \end{class}

        \begin{class}[text width = 6cm]{Food}{-4,-7}
            \inherit{Product}
            \attribute{- calories}
            \operation{+ Food(String, double, int)}
            \operation{+ getCalories() : int}
            \operation{+ setCalories(int) : void}
        \end{class}

        \begin{class}[text width = 6cm]{Drink}{4,-7}
            \inherit{Product}
            \attribute{- size}
            \operation{+ Drink(String, double, String)}
            \operation{+ getSize() : String}
            \operation{+ setSize(String) : void}
        \end{class}
    \end{tikzpicture}
    \caption{继承}
\end{figure}

extends关键字用于指定一个类继承于另一个类,产生继承关系后,子类可以通过super关键字调用父类中的属性和方法,也可以定义子类独有的属性和方法。\\

在创建子类对象时,会先调用父类的构造方法,然后再调用子类的构造方法。因此父类中必须存在一个构造方法,否则将无法创建子类对象。\\

\mybox{麦当劳}

\begin{lstlisting}[language=Java]
public class Product {
    private String name;
    private double price;

    public Product(String name, double price) {
        this.name = name;
        this.price = price;
    }

    public String getName() {
        return name;
    }

    public void setName(String name) {
        this.name = name;
    }

    public double getPrice() {
        return price;
    }

    public void setPrice(double price) {
        this.price = price;
    }
}
\end{lstlisting}

\begin{lstlisting}[language=Java]
public class Food extends Product {
    int calories;

    public Food(String name, double price, int calories) {
        super(name, price);
        this.calories = calories;
    }

    public int getCalories() {
        return calories;
    }

    public void setCalories(int calories) {
        this.calories = calories;
    }
}
\end{lstlisting}

\begin{lstlisting}[language=Java]
public class Drink extends Product {
    private String size;

    public Drink(String name, double price, String size) {
        super(name, price);
        this.size = size;
    }

    public String getSize() {
        return size;
    }

    public void setSize(String size) {
        this.size = size;
    }
}
\end{lstlisting}

\begin{lstlisting}[language=Java]
public class McDonalds {
    public static void main(String[] args) {
        Food food = new Food("Cheeseburger", 5.45, 302);
        Drink drink = new Drink("Coke", 3.7, "Large");

        System.out.printf(
                "Food: %s ($%.2f) %d Kcal\n",
                food.getName(), food.getPrice(), food.getCalories()
        );
        System.out.printf(
                "Drink: %s ($%.2f) %s\n",
                drink.getName(), drink.getPrice(), drink.getSize()
        );
    }
}
\end{lstlisting}

\begin{tcolorbox}
    \mybox{运行结果}
    \begin{verbatim}
Food: Cheeseburger ($5.45) 302 Kcal
Drink: Coke ($3.70) Large
	\end{verbatim}
\end{tcolorbox}

\vspace{0.5cm}

\subsection{重写(Override)}

Object类是所有类的根类,所有的类都直接或者间接地继承自Object类。Object类中包含的方法在其它所有类中都可以使用,例如getClass()、hashCode()、toString()、clone()、equals()等。\\

当直接输出一个对象时,会自动调用该对象的toString()方法,将其以字符串的形式输出。

\vspace{-0.5cm}

\begin{lstlisting}[language=Java]
System.out.println(food);   // Food@41629346
\end{lstlisting}

在没有重写toString()方法的情况下,输出的内容是对象的类名及其哈希码(hash code),但这并不是预期想要的结果。因此,可以重写从父类继承的toString(),以满足程序的需求。\\

在重写方法时,需要使用@Override注解,以便编译器检查该方法是否真的是从父类继承的。\\

\mybox{麦当劳}

\begin{lstlisting}[language=Java]
public class Product {
    private String name;
    private double price;

    public Product(String name, double price) {
        this.name = name;
        this.price = price;
    }

    public String getName() {
        return name;
    }

    public void setName(String name) {
        this.name = name;
    }

    public double getPrice() {
        return price;
    }

    public void setPrice(double price) {
        this.price = price;
    }

    @Override
    public String toString() {
        return String.format("%s ($%.2f)", name, price);
    }
}
\end{lstlisting}

\begin{lstlisting}[language=Java]
public class Food extends Product {
    int calories;

    public Food(String name, double price, int calories) {
        super(name, price);
        this.calories = calories;
    }

    public int getCalories() {
        return calories;
    }

    public void setCalories(int calories) {
        this.calories = calories;
    }

    @Override
    public String toString() {
        return "Food: " + super.toString() + " " + calories + " Kcal";
    }
}
\end{lstlisting}

\begin{lstlisting}[language=Java]
public class Drink extends Product {
    private String size;

    public Drink(String name, double price, String size) {
        super(name, price);
        this.size = size;
    }

    public String getSize() {
        return size;
    }

    public void setSize(String size) {
        this.size = size;
    }

    @Override
    public String toString() {
        return "Drink: " + super.toString() + " " + size;
    }
}
\end{lstlisting}

\begin{lstlisting}[language=Java]
public class McDonalds {
    public static void main(String[] args) {
        Food food = new Food("Cheeseburger", 5.45, 302);
        Drink drink = new Drink("Coke", 3.7, "Large");

        System.out.println(food);
        System.out.println(drink);
    }
}
\end{lstlisting}

\begin{tcolorbox}
	\mybox{运行结果}
	\begin{verbatim}
Food: Cheeseburger ($5.45) 302 Kcal
Drink: Coke ($3.70) Large
	\end{verbatim}
\end{tcolorbox}

\vspace{0.5cm}

\subsection{equals()}

【==】运算符默认比较的是两个对象的地址,如果地址相同则为true,否则为false。\\

equals()方法默认返回地址比较,通过重写equals()方法,可以自定义两个对象的等值比较规则。\\

\mybox{重写equals()}

\begin{lstlisting}[language=Java, title=Dog.java]
public class Dog {
    private String name;
    private int age;
    private String type;

    public Dog(String name, int age, String type) {
        this.name = name;
        this.age = age;
        this.type = type;
    }
    
    /**
     * 自定义规则:实现两个对象的等值比较
     * @param obj - 需要比较的对象
     * @return 比较的结果:相同true,不同false 
     */
    @Override
    public boolean equals(Object obj) {
        // 1. 如果两个对象地址相同,返回true
        if(this == obj) {
            return true;
        }
        
        // 2. 如果obj是null,返回false
        if(obj == null) {
            return false;
        }
        
        // 3. 如果两个对象类型不同,返回false
        if(this.getClass() != obj.getClass()) {
            return false;
        }
        
        // 4. 如果两个对象中的属性全部相同,返回true,否则返回false
        Dog dog = (Dog)obj;
        return this.name.equals(dog.name) 
                && this.age == dog.age 
                && this.type.equals(dog.type);
    }
}
\end{lstlisting}

\begin{lstlisting}[language=Java, title=TestDog.java]
public class TestDog {
    public static void main(String[] args) {
        Dog dog1 = new Dog("狗子", 3, "哈士奇");
        Dog dog2 = new Dog("狗子", 3, "哈士奇");
        
        System.out.println(dog1 == dog2);
        System.out.println(dog1.equals(dog2));
    }
}
\end{lstlisting}

\begin{tcolorbox}
	\mybox{运行结果}
	\begin{verbatim}
false
true
	\end{verbatim}
\end{tcolorbox}

\newpage