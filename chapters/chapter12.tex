\chapter{字符串}

\section{字符串与基本数据类型的转换}

\subsection{基本数据类型转换字符串}

基本数据类型转成字符串,得到的结果是这个数值添加上双引号的样式。 \\

基本数据类型转换字符串有4种方法:

\begin{enumerate}
	\item 利用字符串拼接运算完成:当加号两端有任意一方式字符串的时候,此时会自动的把另一方也转成字符串,完成字符串的拼接。所以,当需要把一个基本数据类型转成字符串的时候,只需要在另一端拼接上一个空的字符串即可。

	\item 【推荐使用】使用字符串的静态方法valueOf()完成。

	\item 借助包装类的实例方法toString()完成。

	\item 借助包装类的静态方法toString()完成。
\end{enumerate}

\vspace{0.5cm}

\mybox{基本数据类型转换字符串}

\begin{lstlisting}[language=Java]
public class BasicToString {
    public static void main(String[] args) {
        int num = 10;
        
        // 1. 利用字符串拼接运算完成
        String s1 = num + "";
        System.out.println(s1);
        
        // 2. 【推荐使用】使用字符串的静态方法valueOf()完成
        String s2 = String.valueOf(num);
        System.out.println(s2);
        
        // 3. 借助包装类的实例方法toString()完成
        String s3 = Integer.valueOf(num).toString();
        System.out.println(s3);
        
        // 4. 借助包装类的静态方法toString()完成
        String s4 = Integer.toString(10);
        System.out.println(s4);
    }
}
\end{lstlisting}

\begin{tcolorbox}
	\mybox{运行结果}
	\begin{verbatim}
10
10
10
10
	\end{verbatim}
\end{tcolorbox}

\subsection{字符串转换基本数据类型}

字符串转换基本数据类型,就是解析出字符串中的内容,转型成对应的基本数据类型的表示。 \\

需要注意的是,基本数据类型转字符串肯定是没有问题的,但是由字符串类型转换基本数据类型的时候,可能会出现问题。字符串中的内容不一定能够转成希望转换的基本数据类型。如果转换失败,会出现NumberFormatException异常。 \\

还需要注意的是,对于整数来说,字符串中如果出现了其它的非数字的字符串,都会导致转整数失败。即便是小数点,也不可以转换,因为这里并没有转成浮点数再转成整数的过程。 \\

字符串转换基本数据类型的方法有2种:

\begin{enumerate}
	\item 使用包装类的静态方法valueOf()完成。
	\item 使用包装类的静态方法parseXXX()完成。
\end{enumerate}

\mybox{字符串转换基本数据类型}

\begin{lstlisting}[language=Java]
public class StringToBasic {
    public static void main(String[] args) {
        // 1. 使用包装类的静态方法valueOf()完成
        Integer num1 = Integer.valueOf("10");
        System.out.println(num1);
        
        // 2. 使用包装类的静态方法parseXXX()完成
        int num2 = Integer.parseInt("10");
        System.out.println(num2);
    }
}
\end{lstlisting}

\begin{tcolorbox}
	\mybox{运行结果}
	\begin{verbatim}
10
10
	\end{verbatim}
\end{tcolorbox}

以上两种方法都可以完成字符串到基本数据类型之间的转换。如果希望直接转成基本数据类型,推荐使用方法2;如果希望得到包装类型,推荐使用方法1。

\newpage

\section{字符串内存分析}

\subsection{字符串内存分析}

字符串是一个引用数据类型,但是字符串的引用与在面向对象部分的引用有一点差别。对于类的对象,是在堆上开辟的空间,而字符串,是在常量池中开辟的空间。 \\

例如\lstinline|String str = "hello"|,"hello"是在常量池中开辟的空间,而str里面存储的其实是常量池中"hello"的地址。当\lstinline|String str = "world"|时,并不是修改了str指向的空间的内容。因为常量池空间特性,一个空间一旦开辟完成了,里面的值是不允许修改的。此时,是在常量池中开辟了一块新的空间,存储了"world",并把这个新的空间的地址赋值给str。 \\

\mybox{字符串内存分析}

\begin{lstlisting}[language=Java]
public class StringMemory {
    public static void main(String[] args) {
        // 第一次使用"hello world"时,常量池中并没有这块内存
        // 此时开辟一块新空间存储"hello world",将其地址赋给s1
        String s1 = "hello world";
        // 再次使用"hello world"时,常量池中已经存在这块内存
        // 此时无需开辟新空间,直接将现有空间地址赋给s2
        String s2 = "hello world";
        // s1和s2都指向"hello world"
        System.out.println(s1 == s2);
    }
}
\end{lstlisting}

\begin{tcolorbox}
	\mybox{运行结果}
	\begin{verbatim}
true
	\end{verbatim}
\end{tcolorbox}

\begin{figure}[H]
	\centering
	\begin{tikzpicture}
		\draw (-4,-5.5) rectangle (-2,-4.5);
		\draw (2,-5.5) rectangle (4,-4.5);
		\draw (-3,-4) node {s1};
		\draw (3,-4) node {s2};

		\draw (-1.5,-7) rectangle (1.5,-6);
		\draw (0,-6.5) node {hello world};
		\draw (0,-7.5) node {常量池};

		\draw[->] (-2,-5) -- (-0.5,-6);
		\draw[->] (2,-5) -- (0.5,-6);
	\end{tikzpicture}
	\caption{常量池}
\end{figure}

String类是Java中用来描述字符串的类,里面也是有构造方法的。通过String类提供的构造方法,实例化的字符串对象是在堆上开辟的空间。在堆空间中,有一个内部维护的属性,指向了常量池中的某一块空间。 \\

\mybox{实例化字符串}

\begin{lstlisting}[language=Java]
public class InstantiateString {
    public static void main(String[] args) {
        // 在堆上开辟了一个String对象的空间,将堆的地址赋给s1
        // 堆空间中有一个内部的属性,指向常量池中的"hello world"
        String s1 = new String("hello world");
        // 在堆上开辟了一个String对象的空间,将堆的地址赋给s2
        // 堆空间中有一个内部的属性,指向常量池中的"hello world"
        String s2 = new String("hello world");
        
        // 此时s1和s2是两块堆空间的地址
        System.out.println(s1 == s2);
        // String类中重写了equals(),实现了比较实际指向常量池中的字符串
        System.out.println(s1.equals(s2));
    }
}
\end{lstlisting}

\begin{tcolorbox}
	\mybox{运行结果}
	\begin{verbatim}
false
true
	\end{verbatim}
\end{tcolorbox}

\begin{figure}[H]
	\centering
	\begin{tikzpicture}
		\draw (-4,-5.5) rectangle (-2,-4.5);
		\draw (2,-5.5) rectangle (4,-4.5);
		\draw (-3,-4) node {s1};
		\draw (3,-4) node {s2};

		\draw (-3,-8) rectangle (3,-6);
		\draw (0,-5.5) node {堆};
		\draw (-2.7,-7.5) rectangle (-0.3,-6.5);
		\draw (-1.5,-7) node {String对象};
		\draw (0.3,-7.5) rectangle (2.7,-6.5);
		\draw (1.5,-7) node {String对象};

		\draw[->] (-2,-5) -- (-1.5,-6.5);
		\draw[->] (2,-5) -- (1.5,-6.5);

		\draw (-1.5,-11) rectangle (1.5,-9);
		\draw (0,-9.5) node {hello world};
		\draw (0,-10.5) node {常量池};

		\draw[->] (-1.5,-7.5) -- (-0.5,-9);
		\draw[->] (1.5,-7.5) -- (0.5,-9);
	\end{tikzpicture}
	\caption{实例化字符串}
\end{figure}

\newpage

\section{字符串构造方法}

\subsection{字符串构造方法}

\begin{table}[H]
	\centering
	\setlength{\tabcolsep}{2mm}{
		\begin{tabular}{|l|l|}
			\hline
			\textbf{构造方法}                         & \textbf{描述}                        \\
			\hline
			String()                                  & 实例化一个空的字符串对象             \\
			\hline
			String(String str)                        & 通过一个字符串,实例化字符串         \\
			\hline
			String(char[] arr)                        & 通过字符数组,实例化字符串           \\
			\hline
			String(char[] arr, int offset, int count) & 通过指定范围的字符数组,实例化字符串 \\
			\hline
			String(byte[] arr)                        & 通过字节数组,实例化字符串           \\
			\hline
			String(byte[] arr, int offset, int count) & 通过指定范围的字节数组,实例化字符串 \\
			\hline
		\end{tabular}
	}
	\caption{字符串构造方法}
\end{table}

\mybox{字符串构造方法}

\begin{lstlisting}[language=Java]
public class StringConstructor {
    public static void main(String[] args) {
        String s1 = new String();
        System.out.println("s1: " + s1);
        
        String s2 = new String("hello");
        System.out.println("s2: " + s2);
        
        String s3 = new String(new char[] {'h', 'e', 'l', 'l', 'o'});
        System.out.println("s3: " + s3);
        
        String s4 = new String(
            new char[] {'h', 'e', 'l', 'l', 'o'}, 1, 3
        );
        System.out.println("s4: " + s4);
        
        String s5 = new String(new byte[] {65, 66, 67});
        System.out.println("s5: " + s5);
        
        String s6 = new String(new byte[] {65, 66, 67}, 0, 2);
        System.out.println("s6: " + s6);
    }
}
\end{lstlisting}

\begin{tcolorbox}
	\mybox{运行结果}
	\begin{verbatim}
s1: 
s2: hello
s3: hello
s4: ell
s5: ABC
s6: AB
	\end{verbatim}
\end{tcolorbox}

\newpage

\section{字符串非静态方法}

\subsection{字符串非静态方法}

因为字符串是常量,任何修改字符串的操作都不会对所修改的字符串造成任何的影响。所有对字符串的修改操作,其实都是实例化了新的字符串对象,在这个新的字符串对象中存储了修改之后的结果,并将这个新的字符串以返回值的形式返回。所以,如果需要得到对一个字符串修改之后的结果,需要接收方法的返回值。 \\

\begin{table}[H]
	\centering
	\setlength{\tabcolsep}{1mm}{
		\begin{tabular}{|l|l|}
			\hline
			\textbf{方法}                                & \textbf{描述}                     \\
			\hline
			.concat(String str)                          & 字符串拼接                        \\
			\hline
			.substring(int beginIndex)                   & 字符串截取                        \\
			\hline
			.substring(int beginIndex, int endIndex)     & 字符串截取                        \\
			\hline
			.replace(char oldChar, char new Char)        & 字符替换                          \\
			\hline
			.replace(CharSequence old, CharSequence new) & 字符序列替换                      \\
			\hline
			.charAt(int index)                           & 获取指定位置的字符                \\
			\hline
			.indexOf(char c)                             & 字符第一次出现的下标              \\
			\hline
			.indexOf(char c, int fromIndex)              & 字符从fromIndex第一次出现的下标   \\
			\hline
			.lastIndexOf(char c)                         & 字符最后一次出现的下标            \\
			\hline
			.lastIndexOf(char c, int fromIndex)          & 字符从fromIndex往前最后出现的下标 \\
			\hline
		\end{tabular}
	}
	\caption{字符串非静态方法}
\end{table}

\mybox{字符串非静态方法}

\begin{lstlisting}[language=Java]
public class StringMethod {
    public static void main(String[] args) {
        // 1. 判断空字符串
        System.out.println("".isEmpty());
        
        // 2. 字符串长度
        System.out.println("Hello World".length());
        
        // 3. 字符串拼接
        System.out.println("Hello".concat("World"));
        
        // 4. 字符串截取
        System.out.println("Hello World".substring(4));
        System.out.println("Hello World".substring(4, 8));
        
        // 5. 字符串替换
        System.out.println("Hello World".replace('l', 'L'));
        System.out.println("Hello World".replace("Hello", "Bye"));
        
        // 6. 获取指定位置字符
        System.out.println("Hello".charAt(1));
        
        // 7. 查询字符位置
        System.out.println("Hello World".indexOf('l'));
        System.out.println("Hello World".indexOf('l', 5));
        System.out.println("Hello World".lastIndexOf('l'));
        System.out.println("Hello World".lastIndexOf('l', 5));
        
        // 8. 去除字符串首位空白字符
        System.out.println("   Hello World   ".trim());
        
        // 9. 大小写转换
        System.out.println("Hello World".toLowerCase());
        System.out.println("Hello World".toUpperCase());
        
        // 10. 判断是否存在子串
        System.out.println("Hello World".contains("llo"));
        
        // 11. 判断是否以指定字符串开头/结尾
        System.out.println("Hello World".startsWith("Hell"));
        System.out.println("Hello World".endsWith("ld"));
        
        // 12. 判断两个字符串内容是否相同
        System.out.println("Hello".equals("Hello"));
        System.out.println("Hello".equalsIgnoreCase("hello"));
        
        // 13. 比较两个字符串大小
        System.out.println("Hello".compareTo("Hall"));
        System.out.println("Hello".compareToIgnoreCase("HELLO"));
    }
}
\end{lstlisting}

\begin{tcolorbox}
	\mybox{运行结果}
	\begin{verbatim}
true
11
HelloWorld
o World
o Wo
HeLLo WorLd
Bye World
e
2
9
9
3
Hello World
hello world
HELLO WORLD
true
true
true
true
true
4
0
	\end{verbatim}
\end{tcolorbox}

\newpage

\section{字符串静态方法}

\subsection{字符串静态方法}

\begin{table}[H]
	\centering
	\setlength{\tabcolsep}{1mm}{
		\begin{tabular}{|l|l|}
			\hline
			\textbf{方法}                                        & \textbf{描述}              \\
			\hline
			.join(CharSequence delimiter, CharSequence elements) & 以指定分隔符拼接若干字符串 \\
			\hline
			.format(String format, Object... args)               & 字符串格式化               \\
			\hline
		\end{tabular}
	}
	\caption{字符串静态方法}
\end{table}

format()方法的占位符如下:

\begin{table}[H]
	\centering
	\setlength{\tabcolsep}{5mm}{
		\begin{tabular}{|c|l|}
			\hline
			\textbf{占位符} & \textbf{描述}                                 \\
			\hline
			\%s             & 代替字符串,\%ns表示凑够n位,如果不够补空格   \\
			\hline
			\%d             & 代替整型数字,\%nd表示凑够n位,如果不够补空格 \\
			\hline
			\%f             & 代替浮点型数字,\%.nf表示保留小数点后n位数字  \\
			\hline
			\%c             & 代替字符                                      \\
			\hline
		\end{tabular}
	}
	\caption{字符串格式化占位符}
\end{table}

\mybox{字符串静态方法}

\begin{lstlisting}[language=Java]
public class StringStaticMethod {
    public static void main(String[] args) {
        // join():字符串拼接
        String[] info = {"2021", "3", "28"};
        String date = String.join("/", info);
        System.out.println(date);

        // format():字符串格式化
        String name = "小灰";
        int age = 18;
        char gender = 'M';
        double height = 178.2;
        System.out.println(String.format(
                "姓名:%s,年龄:%d,性别:%c,身高:%.2f", 
                name, age, gender, height));
    }
}
\end{lstlisting}

\begin{tcolorbox}
	\mybox{运行结果}
	\begin{verbatim}
2021/3/28
姓名:小灰,年龄:18,性别:M,身高:178.20
	\end{verbatim}
\end{tcolorbox}

\newpage

\section{StringBuffer与StringBuilder}

\subsection{StringBuffer / StringBuilder}

字符串是常量,所有操作字符串的方法都不能直接修改字符串本身。如果需要得到修改之后的结果,就需要接收返回值。 \\

StringBuffer和StringBuilder类不是字符串类,而是用来操作字符串的类。在类中维护了一个字符串的属性,这些字符串操作类中的方法,可以直接修改这个属性的值。对于使用方来说,可以不去通过返回值获取操作的结果。

\begin{table}[H]
	\centering
	\setlength{\tabcolsep}{1mm}{
		\begin{tabular}{|l|l|}
			\hline
			\textbf{方法}                           & \textbf{描述}                            \\
			\hline
			构造方法()                              & 实例化空字符串操作类对象                 \\
			\hline
			构造方法(String str)                    & 实例化指定字符串操作类对象               \\
			\hline
			append()                                & 将一个数据拼接到现有的字符串的结尾       \\
			\hline
			insert()                                & 将一个数据插入到字符串的指定位置         \\
			\hline
			delete(int start, int end)              & 删除一个字符串指定范围内的数据           \\
			\hline
			deleteCharAt(int index)                 & 删除指定下标位置的字符                   \\
			\hline
			replace(int start, int end, String str) & 将字符串指定范围内的数据替换成指定字符串 \\
			\hline
			setCharAt(int index, char c)            & 将指定下标位置的字符串替换成新的字符     \\
			\hline
			reverse()                               & 翻转字符串                               \\
			\hline
			toString()                              & 返回一个正在操作的字符串                 \\
			\hline
		\end{tabular}
	}
	\caption{StringBuffer / StringBuilder类常用方法}
\end{table}

\mybox{StringBuffer / StringBuilder}

\begin{lstlisting}[language=Java]
public class TestStringBuffer {
    public static void main(String[] args) {
        StringBuffer sb = new StringBuffer("hello");
        System.out.println(sb);
        
        sb.append("world!");
        System.out.println(sb);
        
        sb.insert(5, ", ");
        System.out.println(sb);
        
        sb.delete(5, 7);
        System.out.println(sb);
        
        sb.replace(0, 5, "Hi");
        System.out.println(sb);
        
        sb.setCharAt(2, 'W');
        System.out.println(sb);
        
        sb.reverse();
        System.out.println(sb);
    }
}
\end{lstlisting}

\begin{tcolorbox}
	\mybox{运行结果}
	\begin{verbatim}
hello
helloworld!
hello, world!
helloworld!
Hiworld!
HiWorld!
!dlroWiH
	\end{verbatim}
\end{tcolorbox}

\subsection{StringBuffer与StringBuilder的区别}

StringBuffer和StringBuilder类从功能上来讲是一模一样的,但是他们还是有区别的:

\begin{itemize}
	\item StringBuffer是线程安全的:当处于多线程的环境中,多个线程同时操作这个对象,此时使用StringBuffer类。

	\item StringBuilder是线程不安全的:当没有处于多线程的环境中,只有一个线程来操作这个对象,此时使用StringBuilder类。
\end{itemize}

但凡是涉及到字符串操作的使用场景,特别是在循环中对字符串进行的操作,一定不要使用字符串的方法,而要使用StringBuffer或者StringBuilder的方法来做。 \\

由于字符串本身是不可变的,所以String类所有的修改操作,其实都是在方法内实例化了一个新的字符串对象,存储修改之后的新的字符串的地址,返回这个新的字符串。如果操作比较频繁,就意味着有大量的临时字符串被实例化、被销毁,效率极低。 \\

StringBuffer和StringBuilder类不同,它们在内部维护了一个字符数组,所有的操作都是围绕这个字符数组进行的。当需要转成字符串的时候,才会调用toString()进行转换。当频繁用到字符串操作的时候,没有中间的临时字符串出现,效率较高。 \\

\mybox{比较String、StringBuffer、StringBuilder类的效率}

\begin{lstlisting}[language=Java]
public class CompareEfficiency {
    public static void main(String[] args) {
        final int CNT = 100000;
        
        String str = new String();
        StringBuffer stringBuffer = new StringBuffer();
        StringBuilder stringBuilder = new StringBuilder();
        long start, end;
        
        // String拼接
        start = System.currentTimeMillis();
        for(int i = 0; i < CNT; i++) {
            str += i;
        }
        end = System.currentTimeMillis();
        System.out.println("String拼接:" + (end - start));
        
        // StringBuffer拼接
        start = System.currentTimeMillis();
        for(int i = 0; i < CNT; i++) {
            stringBuffer.append(i);
        }
        end = System.currentTimeMillis();
        System.out.println("StringBuffer拼接:" + (end - start));
        
        // StringBuilder拼接
        start = System.currentTimeMillis();
        for(int i = 0; i < CNT; i++) {
            stringBuilder.append(i);
        }
        end = System.currentTimeMillis();
        System.out.println("StringBuilder拼接:" + (end - start));
    }
}
\end{lstlisting}

\begin{tcolorbox}
	\mybox{运行结果}
	\begin{verbatim}
String拼接:4326
StringBuffer拼接:2
StringBuilder拼接:2
	\end{verbatim}
\end{tcolorbox}

\newpage