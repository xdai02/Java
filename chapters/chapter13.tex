\chapter{正则表达式}

\section{正则表达式}

\subsection{正则表达式(Regular Expression)}

正则表达式不是Java所有的,它是一套独立的、自成体系的知识点。在很多语言中都有对正则的使用。 \\

正则表达式是用来做字符串的校验、匹配、验证一个字符串是否与指定的规则匹配。 \\

在很多的语言中,都在匹配的基础上添加了其它的功能。例如Java,在匹配的基础上还添加了删除、替换等功能。 \\

\mybox{使用与不使用正则表达式的区别}

\begin{lstlisting}[language=Java]
/**
 * 验证一个字符串是否是一个合法的账号
 * 规则:
 *      1. 纯数字组成
 *      2. 不能以0开头
 *      3. 长度[6, 11]
 */
public class CheckAccount {
    public static void main(String[] args) {
        // 不使用正则表达式
        System.out.println(validateAccount("2513276112"));
        System.out.println(validateAccount("012.3"));
        
        // 使用正则表达式
        System.out.println(validateAccountWithRegex("h3ll0"));
        System.out.println(validateAccountWithRegex("28368346"));
    }
    
    public static boolean validateAccount(String account) {
        // 1. 纯数字组成
        int len = account.length();
        for(int i = 0; i < len; i++) {
            if(account.charAt(i) < '0' || account.charAt(i) > '9') {
                return false;
            }
        }
        
        // 2. 不能以0开头
        if(account.startsWith("0")) {
            return false;
        }
        
        // 3. 长度[6, 11]
        if(len < 6 || len > 11) {
            return false;
        }
        return true;
    }
    
    public static boolean validateAccountWithRegex(String account) {
        // 第1位数字为[1-9],后面[0-9]可重复5-10次
        return account.matches("[1-9]\\d{5,10}");
    }
}
\end{lstlisting}

\begin{tcolorbox}
	\mybox{运行结果}
	\begin{verbatim}
true
false
false
true
	\end{verbatim}
\end{tcolorbox}

\newpage

\section{匹配规则}

\subsection{匹配规则}

正则表达式的匹配规则是逐个字符进行匹配,判断是否和正则表达式中定义的规则一致。 \\

boolean matches(String regex)是String类中的非静态方法,使用字符串对象调用这个方法,参数是一个正则表达式。

\subsection{元字符(metacharacter)}

正则表达式中的元字符如下:

\begin{itemize}
	\item \^{}:匹配一个字符串的开头,在Java的正则匹配中可以省略不写。

	\item \$:匹配一个字符串的结尾,在Java的正则匹配中可以省略不写。

	\item $ [] $:匹配一位字符。例如:[abc]表示这一位字符可以是a或b或c。[a-z]表示这一位字符可以是[a, z]范围内的任意字符。[a-zABC]表示这一位字符可以是[a, z]范围内的任意字符,或者A或B或C。[a-zA-Z]表示这一位字符可以是任意的大小写字母。[\^{}abc]表示这一位字符可以是除了a、b、c以外的任意字符。

	\item $ \backslash $:转义字符。使得某些特殊字符成为普通字符,可以进行规则的指定。使得某些普通字符变得具有特殊含义。由于正则表达式在Java中是需要写在一个字符串中,而字符串中的【$ \backslash $】也是一个转义字符,因此Java中写正则表达式的时候,转义字符需要使用【$ \backslash\backslash $】。

	\item $ \backslash d $:匹配所有的数字,等同于[0-9]。

	\item $ \backslash D $:匹配所有的非数字,等同于[\^{}0-9]。

	\item $ \backslash w $:匹配所有的单词字符,等同于[a-zA-Z0-9\_]。

	\item $ \backslash W $:匹配所有的非单词字符,等同于[\^{}a-zA-Z0-9\_]。

	\item .:通配符,可以匹配一个任意的字符。

	\item +:前面的一位或者一组字符,连续出现了一次或多次。

	\item ?:前面的一位或者一组字符,连续出现了一次或零次。

	\item *:前面的一位或者一组字符,连续出现了零次、一次或多次。

	\item \{\}:对前面的一位或者一组字符出现次数的精准匹配。\{m\}表示前面的一位或者一组字符连续出现了m次。\{m,\}表示前面的一位或者一组字符连续出现了至少m次。\{m,n\}表示前面的一位或者一组字符连续出现了至少m次,最多n次。

	\item ():分组,把某些连续的字符视为一个整体对待。

	\item |:作用于整体或者是一个分组,表示匹配的内容可以是任意的一个部分。abc|123表示可以是abc,也可以是123。
\end{itemize}

\mybox{验证合法性}

\begin{lstlisting}[language=Java]
public class Verification {
    public static void main(String[] args) {
        // 1. 验证QQ账号:长度5-11,首位不为0
        System.out.println("2513276112".matches("[1-9]\\d{4,10}"));
        
        // 2. 验证QQ邮箱:QQ号码@qq.com
        System.out.println("2513276112@qq.com".matches(
            "[1-9]\\d{4,10}@qq\\.com")
        );
        
        // 3. 验证手机号
        System.out.println("13671712345".matches("1[356789]\\d{9}"));
        
        // 4. 验证固定电话:区号(3-4位)-电话号码(8位)
        System.out.println("021-55031234".matches("\\d{3,4}-\\d{8}"));
        
        // 5. 验证126或163邮箱:邮箱名(4-12位有效字符)@126/163.com
        System.out.println("admin123@163.com".matches(
            "\\w{4,12}@(126|163)\\.com")
        );
    }
}
\end{lstlisting}

\begin{tcolorbox}
	\mybox{运行结果}
	\begin{verbatim}
true
true
true
true
true
	\end{verbatim}
\end{tcolorbox}

\mybox{隐藏手机号中间4位}

\begin{lstlisting}[language=Java]
public class Mobile {
    public static void main(String[] args) {
        // $1表示获取第1个分组的值
        // $3表示获取第3个分组的值
        System.out.println("13671712345".replaceAll(
            "(\\d{3})(\\d{4})(\\d{3})", 
            "$1****$3"));
    }
}
\end{lstlisting}

\begin{tcolorbox}
	\mybox{运行结果}
	\begin{verbatim}
136****2345
	\end{verbatim}
\end{tcolorbox}