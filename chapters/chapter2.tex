\chapter{数据类型}

\section{变量}

\subsection{变量(Variable)}

Java是一种强类型的语言,任何数据都有一个确定的类型。 \\

变量是计算机中一块特定的内存空间,由一个或多个连续的字节组成,不同数据存入具有不同内存地址的空间,相互独立,通过变量名可以简单快速地找到在内存中存储的数据。 \\

变量名需要符合以下的要求:

\begin{enumerate}
	\item 由字母、数字和下划线组成,第一个字符必须为字母或下划线。
	\item 不能包含除【\_】以外的任何特殊字符,如【\%】、【\#】等。
	\item 不可以使用保留字或关键字。
	\item 准确、顾名思义,不要使用汉语拼音。
\end{enumerate}

关键字是编程语言内置的一些名称,具有特殊的用处和意义。

\begin{table}[H]
	\centering
	\setlength{\tabcolsep}{5mm}{
		\begin{tabular}{|c|c|c|c|c|}
			\hline
			abstract & do      & implements & protected & throws    \\
			\hline
			boolean  & double  & import     & public    & transient \\
			\hline
			break    & else    & instanceof & return    & true      \\
			\hline
			byte     & extends & int        & short     & try       \\
			\hline
			case     & false   & interface  & static    & void      \\
			\hline
			catch    & final   & long       & strict    & volatile  \\
			\hline
			char     & finally & native     & super     & while     \\
			\hline
			class    &         &            &           &           \\
			\hline
		\end{tabular}
	}
	\caption{关键字}
\end{table}

\subsection{数据类型}

Java中变量主要有三大类型:

\begin{enumerate}
	\item 整型
	      \begin{itemize}
		      \item 字节型byte
		      \item 短整型short
		      \item 整型int
		      \item 长整型long
	      \end{itemize}

	\item 浮点型
	      \begin{itemize}
		      \item 单精度浮点型float
		      \item 双精度浮点型double
	      \end{itemize}

	\item 字符型char

	\item 布尔型boolean
\end{enumerate}

\begin{table}[H]
	\centering
	\setlength{\tabcolsep}{5mm}{
		\begin{tabular}{|c|c|c|}
			\hline
			\textbf{数据类型} & \textbf{位数} & \textbf{取值范围}           \\
			\hline
			int               & 32            & $ -2^{31} \sim 2^{31} - 1 $ \\
			\hline
			float             & 32            & $ -3.4E38 \sim 3.4E38 $     \\
			\hline
			double            & 64            & $ -1.7E308 \sim 1.7E308 $   \\
			\hline
			char              & 8             & $ -128 ~ 127 $              \\
			\hline
			boolean           & 1             & true / false                \\
			\hline
		\end{tabular}
	}
	\caption{不同数据类型的取值范围}
\end{table}

\newpage

\section{初始化}

\subsection{初始化(Initialization)}

变量可以在定义时初始化,也可以在定义后初始化。 \\

在编程中,【=】不是数学中的“等于”符号,而是表示“赋值”,即将【=】右边的值赋给左边的变量。

\begin{lstlisting}[language=Java]
int n = 10;
double wage = 8232.56;
\end{lstlisting}

\subsection{常量(Constant)}

常量是一个固定值,在程序执行期间不会改变,即在定义后不可修改。常量可以是任何的基本数据类型,比如整数常量、浮点常量、字符常量。 \\

\mybox{常量}
\begin{lstlisting}[language=Java]
public class Contant {
    public static void main(String[] args) {
        final double PI = 3.14159;
        PI = 4;
    }
}
\end{lstlisting}

\begin{tcolorbox}
	\mybox{运行结果} \\
	\textcolor{red}{The final local variable PI cannot be assigned.}
\end{tcolorbox}

\section{算术运算符}

\subsection{四则运算}

\begin{table}[H]
	\centering
	\setlength{\tabcolsep}{5mm}{
		\begin{tabular}{|c|c|c|}
			\hline
			\textbf{数学符号} & \textbf{Java符号} & \textbf{含义} \\
			\hline
			$ + $             & +                 & 加法          \\
			\hline
			$ - $             & -                 & 减法          \\
			\hline
			$ \times $        & *                 & 乘法          \\
			\hline
			$ \div $          & /                 & 除法          \\
			\hline
			$ \dots\dots $    & \%                & 取模          \\
			\hline
		\end{tabular}
	}
	\caption{四则运算}
\end{table}

Java中除法【/】的意义与数学中不同:

\begin{enumerate}
	\item 当相除的两个运算数都为整型,则运算结果为两个数进行除法运算后的整数部分,例如21 / 5的结果为4。

	\item 如果两个运算数其中至少一个为浮点型,则运算结果为浮点型,如21 / 5.0的结果为4.2。
\end{enumerate}

取模(modulo)【\%】表示求两个数相除之后的余数,如22 \% 3的结果为1;4 \% 7的结果为4。

\subsection{复合赋值运算符}

\begin{table}[H]
	\centering
	\setlength{\tabcolsep}{5mm}{
		\begin{tabular}{|c|c|}
			\hline
			\textbf{运算符} & \textbf{描述}           \\
			\hline
			+=              & a += b等价于a = a + b   \\
			\hline
			-=              & a -= b等价于a = a - b   \\
			\hline
			*=              & a *= b等价于a = a * b   \\
			\hline
			/=              & a /= b等价于a = a / b   \\
			\hline
			\%=             & a \%= b等价于a = a \% b \\
			\hline
		\end{tabular}
	}
	\caption{复合赋值运算符}
\end{table}

\newpage

\section{输入输出函数}

\subsection{System.out.println()}

System.out.println()的功能是向屏幕输出指定格式的字符串内容。通过【+】运算符可以连接两个字符串。

\mybox{字符串连接}
\begin{lstlisting}[language=Java]
public class StringConcatenation {
	public static void main(String[] args) {
		System.out.println("Hello" + "World");
		System.out.println("Hello" + 2);
		System.out.println("Hello" + 2 + 3);
		System.out.println(2 + 3 + "Hello");
	}
}
\end{lstlisting}

\begin{tcolorbox}
	\mybox{运行结果} \\
	HelloWorld \\
	Hello2 \\
	Hello23 \\
	5Hello
\end{tcolorbox}

\subsection{转义字符}

在一个字符串描述的过程中,有可能会有一些特殊字符的信息。

\begin{table}[H]
	\centering
	\setlength{\tabcolsep}{5mm}{
		\begin{tabular}{|c|c|}
			\hline
			\textbf{转义字符}      & \textbf{描述}                        \\
			\hline
			\lstinline|\\| & 表示一个反斜杠\lstinline|\| \\
			\hline
			\lstinline|\'| & 表示一个单引号\lstinline|'| \\
			\hline
			\lstinline|\"| & 表示一个双引号\lstinline|"| \\
			\hline
			\lstinline|\n| & 换行                                 \\
			\hline
			\lstinline|\t| & 横向制表符                           \\
			\hline
		\end{tabular}
	}
	\caption{转义字符}
\end{table}

\mybox{转义字符}
\begin{lstlisting}[language=Java]
public class EscapeCharacter {
	public static void main(String[] args) {
		System.out.print("全球最大同性交友网站\n");
		System.out.println("\'https://github.com\'");
	}
}
\end{lstlisting}

\begin{tcolorbox}
	\mybox{运行结果} \\
	全球最大同性交友网站 \\
	'https://github.com'
\end{tcolorbox}

\subsection{Scanner}

通过Scanner类可以获取用户的输入,使用Scanner类需要导入java.util.Scanner。根据实例化的Scanner类的对象,调用next()和nextLine()可以获取输入的字符串。 \\

使用完Scanner类的对象后,需要关闭输入流。 \\

\mybox{计算长方形面积}
\begin{lstlisting}[language=Java]
import java.util.Scanner;

public class CircleArea {
	public static void main(String[] args) {
		Scanner scanner = new Scanner(System.in);
		double length, width;
		double area;
		
		System.out.print("输入长度:");
		length = scanner.nextDouble();
		System.out.print("输入宽度:");
		width = scanner.nextDouble();
		area = length * width;
		
		System.out.println(String.format("面积 = %.2f", area));
		scanner.close();
	}
}
\end{lstlisting}

\begin{tcolorbox}
	\mybox{运行结果} \\
	输入长度:20 \\
	输入宽度:30 \\
	面积 = 600.00
\end{tcolorbox}

\vspace{0.5cm}
\mybox{计算圆面积}
\begin{lstlisting}[language=Java]
import java.util.Scanner;

public class CircleArea {
	public static void main(String[] args) {
		Scanner scanner = new Scanner(System.in);
		final double PI = 3.14159;
		double r;
		double area;
		
		System.out.print("输入半径:");
		r = scanner.nextDouble();
		area = PI * r * r;
		
		System.out.println(String.format("面积 = %.2f", area));
		scanner.close();
	}
}
\end{lstlisting}

\begin{tcolorbox}
	\mybox{运行结果} \\
	输入半径:5 \\
	面积 = 78.539750
\end{tcolorbox}

\vspace{0.5cm}
\mybox{逆序三位数}
\begin{lstlisting}[language=Java]
import java.util.Scanner;

public class Reverse {
	public static void main(String[] args) {
		Scanner scanner = new Scanner(System.in);
		int num;
		int a, b, c;

		System.out.print("输入一个三位数:");
		num = scanner.nextInt();

		a = num / 100;
		b = num / 10 % 10;
		c = num % 10;

		System.out.println("逆序:" + (c*100 + b*10 + a));
		scanner.close();
	}
}
\end{lstlisting}

\begin{tcolorbox}
	\mybox{运行结果} \\
	输入一个正三位数:520 \\
	逆序:25
\end{tcolorbox}

\newpage

\section{类型转换}

\subsection{类型转换}

类型转换是把变量从一种类型转换为另一种数据类型。类型转换可以是隐式的,由编译器自动执行,也可以是显式的,通过使用强制类型转换运算符来指定。在有需要类型转换时都用上强制类型转换运算符是一种良好的编程习惯。 \\

\mybox{隐式类型转换}
\begin{lstlisting}[language=Java]
public class Implitcit {
	public static void main(String[] args) {
		int a = 1;
		double b = a;
		System.out.println("b = " + b);
	}
}
\end{lstlisting}

\begin{tcolorbox}
	\mybox{运行结果} \\
	b = 2
\end{tcolorbox}

\vspace{0.5cm}
\mybox{显式类型转换}
\begin{lstlisting}[language=Java]
public class Explicit {
	public static void main(String[] args) {
		int sum = 821;
		int num = 10;
		double average = (double)sum / num;
		System.out.println(String.format("average = %.2f", average));
	}
}
\end{lstlisting}

\begin{tcolorbox}
	\mybox{运行结果} \\
	82.10
\end{tcolorbox}

\newpage