\chapter{数据类型}

\section{变量}

\subsection{变量(Variable)}

Java是一种强类型的语言,任何数据都有一个确定的类型。 \\

变量是计算机中一块特定的内存空间,由一个或多个连续的字节组成,不同数据存入具有不同内存地址的空间,相互独立,通过变量名可以简单快速地找到在内存中存储的数据。 \\

变量名需要符合以下的要求:

\begin{enumerate}
	\item 由字母、数字和下划线组成,第一个字符必须为字母或下划线。
	\item 不能包含除【\_】以外的任何特殊字符,如【\%】、【\#】等。
	\item 不可以使用保留字或关键字。
	\item 准确、顾名思义,不要使用汉语拼音。
\end{enumerate}

关键字是编程语言内置的一些名称,具有特殊的用处和意义。

\begin{table}[H]
	\centering
	\setlength{\tabcolsep}{5mm}{
		\begin{tabular}{|c|c|c|c|c|}
			\hline
			abstract & do      & implements & protected & throws    \\
			\hline
			boolean  & double  & import     & public    & transient \\
			\hline
			break    & else    & instanceof & return    & true      \\
			\hline
			byte     & extends & int        & short     & try       \\
			\hline
			case     & false   & interface  & static    & void      \\
			\hline
			catch    & final   & long       & strict    & volatile  \\
			\hline
			char     & finally & native     & super     & while     \\
			\hline
			class    &         &            &           &           \\
			\hline
		\end{tabular}
	}
	\caption{关键字}
\end{table}

\subsection{数据类型}

Java中变量主要有三大类型:

\begin{enumerate}
	\item 整型
	      \begin{itemize}
		      \item 字节型byte
		      \item 短整型short
		      \item 整型int
		      \item 长整型long
	      \end{itemize}

	\item 浮点型
	      \begin{itemize}
		      \item 单精度浮点型float
		      \item 双精度浮点型double
	      \end{itemize}

	\item 字符型char

	\item 布尔型boolean
\end{enumerate}

\begin{table}[H]
	\centering
	\setlength{\tabcolsep}{5mm}{
		\begin{tabular}{|c|c|c|}
			\hline
			\textbf{数据类型} & \textbf{位数} & \textbf{取值范围}           \\
			\hline
			int               & 32            & $ -2^{31} \sim 2^{31} - 1 $ \\
			\hline
			float             & 32            & $ -3.4E38 \sim 3.4E38 $     \\
			\hline
			double            & 64            & $ -1.7E308 \sim 1.7E308 $   \\
			\hline
			char              & 8             & $ -128 ~ 127 $              \\
			\hline
			boolean           & 1             & true / false                \\
			\hline
		\end{tabular}
	}
	\caption{不同数据类型的取值范围}
\end{table}

\newpage

\section{初始化}

\subsection{初始化(Initialization)}

变量可以在定义时初始化,也可以在定义后初始化。 \\

在编程中,【=】不是数学中的“等于”符号,而是表示“赋值”,即将【=】右边的值赋给左边的变量。

\begin{lstlisting}[language=Java]
int n = 10;
double wage = 8232.56;
\end{lstlisting}

\subsection{常量(Constant)}

常量是一个固定值,在程序执行期间不会改变,即在定义后不可修改。常量可以是任何的基本数据类型,比如整数常量、浮点常量、字符常量。 \\

\mybox{常量}
\begin{lstlisting}[language=Java]
public class Contant {
    public static void main(String[] args) {
        final double PI = 3.14159;
        PI = 4;
    }
}
\end{lstlisting}

\begin{tcolorbox}
	\mybox{运行结果} \\
	\textcolor{red}{The final local variable PI cannot be assigned.}
\end{tcolorbox}

\section{算术运算符}

\subsection{四则运算}

\begin{table}[H]
	\centering
	\setlength{\tabcolsep}{5mm}{
		\begin{tabular}{|c|c|c|}
			\hline
			\textbf{数学符号} & \textbf{Java符号} & \textbf{含义} \\
			\hline
			$ + $             & +               & 加法          \\
			\hline
			$ - $             & -               & 减法          \\
			\hline
			$ \times $        & *               & 乘法          \\
			\hline
			$ \div $          & /               & 除法          \\
			\hline
			$ \dots\dots $    & \%              & 取模          \\
			\hline
		\end{tabular}
	}
	\caption{四则运算}
\end{table}

Java中除法【/】的意义与数学中不同:

\begin{enumerate}
	\item 当相除的两个运算数都为整型,则运算结果为两个数进行除法运算后的整数部分,例如21 / 5的结果为4。

	\item 如果两个运算数其中至少一个为浮点型,则运算结果为浮点型,如21 / 5.0的结果为4.2。
\end{enumerate}

取模(modulo)【\%】表示求两个数相除之后的余数,如22 \% 3的结果为1;4 \% 7的结果为4。

\subsection{复合赋值运算符}

\begin{table}[H]
	\centering
	\setlength{\tabcolsep}{5mm}{
		\begin{tabular}{|c|c|}
			\hline
			\textbf{运算符} & \textbf{描述}                                        \\
			\hline
			+=              & a += b等价于a = a + b \\
			\hline
			-=              & \lstinline|a -= b|等价于\lstinline|a = a - b| \\
			\hline
			*=              & \lstinline|a *= b|等价于\lstinline|a = a * b| \\
			\hline
			/=              & \lstinline|a /= b|等价于\lstinline|a = a / b| \\
			\hline
			\%=             & \lstinline|a %= b|等价于\lstinline|a = a % b| \\
			\hline
		\end{tabular}
	}
	\caption{复合赋值运算符}
\end{table}

\newpage