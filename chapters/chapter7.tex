\chapter{继承}

\section{继承}

\subsection{继承(Inheritance)}

继承是面向对象的三大特征之一,程序中的继承是类与类之间的特征和行为的一种赠予或获取。两个类之间的继承必须满足“is a”的关系。子类继承自父类,父类也称基类或超类,子类也称派生类。

\begin{figure}[H]
	\centering
	\begin{tikzpicture}
		\begin{class}{Animal}{0,0}
			\attribute{- age}
			\attribute{- gender}
			\operation{+ eat() : void}
			\operation{+ sleep() : void}
		\end{class}

		\begin{class}{Dog}{-5,-5}
			\inherit{Animal}
			\attribute{- type}
			\operation{+ bite() : void}
		\end{class}

		\begin{class}{Cat}{5,-5}
			\inherit{Animal}
			\attribute{- hairColor}
			\operation{+ mewing() : void}
		\end{class}
	\end{tikzpicture}
	\caption{继承}
\end{figure}

产生继承关系后,子类可以使用父类中的属性和方法,也可以定义子类独有的属性和方法。

\vspace{-0.5cm}

\begin{lstlisting}[language=Java]
class subclass extends superclass {
    // code
}
\end{lstlisting}

继承时通常使用public类型。当一个类public继承于父类时,父类的public成员也是子类的public成员,父类的protected成员也是子类的protected成员,父类的private成员不能被继承。\\

继承的好处是可以提高代码的复用性、提高代码的拓展性。\\

父类中以下内容是不可以被继承的:

\begin{enumerate}
	\item 构造方法:构造方法是为了创建当前类的对象的。
	\item 私有成员:私有成员只能在当前的类中使用。
	\item 跨包子类:默认权限的属性、方法,不可以继承给跨包的子类。
\end{enumerate}

\vspace{0.5cm}

\subsection{访问权限修饰符}

访问权限修饰符,就是修饰类、属性的访问权限。

\begin{table}[H]
	\centering
	\setlength{\tabcolsep}{5mm}{
		\begin{tabular}{|c|c|c|c|c|}
			\hline
			\textbf{访问权限修饰符}  & \textbf{当前类} & \textbf{同包其它类} & \textbf{跨包子类} & \textbf{跨包其它类} \\
			\hline
			\textbf{private}         & \checkmark      & \XSolidBrush        & \XSolidBrush      & \XSolidBrush        \\
			\hline
			\textbf{default(不写)} & \checkmark      & \checkmark          & \XSolidBrush      & \XSolidBrush        \\
			\hline
			\textbf{protected}       & \checkmark      & \checkmark          & \checkmark        & \XSolidBrush        \\
			\hline
			\textbf{public}          & \checkmark      & \checkmark          & \checkmark        & \checkmark          \\
			\hline
		\end{tabular}
	}
	\caption{访问权限修饰符}
\end{table}

\vspace{0.5cm}

\subsection{super}

使用super可以调用父类的方法。子类对象在实例化的时候,需要先实例化从父类继承到的部分,此时默认调用父类中的无参构造方法。\\

如果父类中没有无参构造方法,对所有的子类对象实例化都会造成影响,导致子类对象无法实例化。\\

解决方法有两种:

\begin{enumerate}
	\item 为父类添加无参构造方法。
	\item 在子类的构造方法中,使用super()调用父类中的有参构造函数。
\end{enumerate}

\mybox{继承}

\begin{lstlisting}[language=Java, title=Animal.java]
public class Animal {
    private String name;
    private int age;

    public Animal() {
        this.name = "";
        this.age = 0;
    }

    public Animal(String name, int age) {
        this.name = name;
        this.age = age;
    }
    
    public void eat() {
        System.out.println("吃饭");
    }

    public void sleep() {
        System.out.println("睡觉");
    }
}
\end{lstlisting}

\begin{lstlisting}[language=Java, title=Dog.java]
public class Dog extends Animal {
    private String type;

    public Dog(String name, int age, String type) {
        super(name, age);
        this.type = type;
    }
    
    public void bite() {
        System.out.println("咬人");
    }
}
\end{lstlisting}

\begin{lstlisting}[language=Java, title=TestDog.java]
public class TestDog {
    public static void main(String[] args) {
        Dog dog = new Dog("狗子", 3, "哈士奇");
        dog.eat();
        dog.sleep();
        dog.bite();
    }
}
\end{lstlisting}

\begin{tcolorbox}
	\mybox{运行结果}
	\begin{verbatim}
吃饭
睡觉
咬人
	\end{verbatim}
\end{tcolorbox}

\newpage

\section{重写}

\subsection{toString()}

Object类是Java中类层次的根类,所有的类都直接或者间接地继承自Object类。Object类的toString()方法返回一个字符串,该字符串由类名、标记符【@】和此对象的哈希码的无符号十六进制表示组成。\\

\mybox{toString()}

\begin{lstlisting}[language=Java, title=Dog.java]
public class Dog extends Animal {
    private String name;
    private String type;
    
    public Dog(String name, String type) {
        this.name = name;
        this.type = type;
    }   
}
\end{lstlisting}

\begin{lstlisting}[language=Java, title=TestDog.java]
public class TestDog {
    public static void main(String[] args) {
        Dog dog = new Dog("狗子", "哈士奇");
        System.out.println(dog);
    }
}
\end{lstlisting}

\begin{tcolorbox}
	\mybox{运行结果}
	\begin{verbatim}
Dog@28a418fc
	\end{verbatim}
\end{tcolorbox}

\vspace{0.5cm}

\subsection{重写(Override)}

子类可以继承到父类中的属性和方法,但是有些方法,子类的实现与父类的方法可能实现的不同。当父类提供的方法已经不能满足子类的需求时,子类中可以定义与父类相同的方法。此时,子类方法完成对父类方法的覆盖,称为重写。\\

重写方法需要注意以下几点:

\begin{enumerate}
	\item 方法名字必须和父类方法名字相同。
	\item 参数列表必须和父类一致。
	\item 方法的访问权限需要大于等于父类方法的访问权限。
	\item 方法的返回值类型需要小于等于父类方法的返回值类型。
\end{enumerate}

@Override是一个注解,用于进行重写前的校验,校验一个方法是否是一个重写的方法,如果不是重写的方法,会直接报错。\\

@Override只是对方法进行重写的验证,与这个方法是不是重写方法无关。在写重写方法的时候,这个注解最好加上。\\

\mybox{重写toString()}

\begin{lstlisting}[language=Java, title=Dog.java]
public class Dog {
    private String name;
    private String type;
    
    public Dog(String name, String type) {
        this.name = name;
        this.type = type;
    }

    @Override
    public String toString() {
        return "我叫" + this.name + ",我是一只" + this.type;
    }
}
\end{lstlisting}

\begin{lstlisting}[language=Java, title=TestDog.java]
public class TestDog {
    public static void main(String[] args) {
        Dog dog = new Dog("狗子", "哈士奇");
        System.out.println(dog);
    }
}
\end{lstlisting}

\begin{tcolorbox}
	\mybox{运行结果}
	\begin{verbatim}
我叫狗子,我是一只哈士奇
	\end{verbatim}
\end{tcolorbox}

\vspace{0.5cm}

\subsection{equals()}

【==】运算符默认比较的是两个对象的地址,如果地址相同则为true,否则为false。\\

equals()方法默认返回地址比较,通过重写equals()方法,可以自定义两个对象的等值比较规则。\\

\mybox{重写equals()}

\begin{lstlisting}[language=Java, title=Dog.java]
public class Dog {
    private String name;
    private int age;
    private String type;

    public Dog(String name, int age, String type) {
        this.name = name;
        this.age = age;
        this.type = type;
    }
    
    /**
     * 自定义规则:实现两个对象的等值比较
     * @param obj - 需要比较的对象
     * @return 比较的结果:相同true,不同false 
     */
    @Override
    public boolean equals(Object obj) {
        // 1. 如果两个对象地址相同,返回true
        if(this == obj) {
            return true;
        }
        
        // 2. 如果obj是null,返回false
        if(obj == null) {
            return false;
        }
        
        // 3. 如果两个对象类型不同,返回false
        if(this.getClass() != obj.getClass()) {
            return false;
        }
        
        // 4. 如果两个对象中的属性全部相同,返回true,否则返回false
        Dog dog = (Dog)obj;
        return this.name.equals(dog.name) 
                && this.age == dog.age 
                && this.type.equals(dog.type);
    }
}
\end{lstlisting}

\begin{lstlisting}[language=Java, title=TestDog.java]
public class TestDog {
    public static void main(String[] args) {
        Dog dog1 = new Dog("狗子", 3, "哈士奇");
        Dog dog2 = new Dog("狗子", 3, "哈士奇");
        
        System.out.println(dog1 == dog2);
        System.out.println(dog1.equals(dog2));
    }
}
\end{lstlisting}

\begin{tcolorbox}
	\mybox{运行结果}
	\begin{verbatim}
false
true
	\end{verbatim}
\end{tcolorbox}

\newpage