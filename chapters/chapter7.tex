\chapter{封装}

\section{面向过程与面向对象}

\subsection{面向过程(Procedure Oriented)}

面向过程是一种以过程为中心的编程思想,以什么正在发生为主要目标进行编程,分析出解决问题所需要的步骤,然后用函数把这些步骤一步一步实现,使用的时候一个一个依次调用。\\

C语言就是一种面向过程的编程语言,但是面向过程的缺陷是数据和函数并不完全独立,使用两个不同的实体表示信息及其操作。\\

\subsection{面向对象(Object Oriented)}

面向对象是相对于面向过程来讲的,面向对象方法把相关的数据和方法组织为一个整体来看待,从更高的层次来进行系统建模,更贴近事物的自然运行模式。\\

在面向对象中,把构成问题的事物分解成各个对象,建立对象的目的不是为了完成一个步骤,而是为了描叙某个事物在整个解决问题的步骤中的行为。\\

Java、C++、Python等都是面向对象的编程语言,面向对象的优势在于只是用一个实体就能同时表示信息及其操作。\\

面向对象三大特性:

\begin{enumerate}
	\item 封装(encapsulation):数据和代码捆绑,避免外界干扰和不确定性访问。
	\item 继承(inheritance):让某种类型对象获得另一类型对象的属性和方法。
	\item 多态(polymorphism):同一事物表现出不同事物的能力。
\end{enumerate}

\newpage

\section{类与对象}

\subsection{类与对象}

类(class)表示同一类具有相同特征和行为的对象的集合,类定义了对象的属性和方法。\\

对象(object)是类的实例,对象拥有属性和方法。\\

类的设计需要使用关键字class,类名是一个标识符,遵循大驼峰命名法。类中可以包含属性和方法。其中,属性通过变量表示,又称实例变量;方法用于描述行为,又称实例方法。\\

通过关键字new进行对象的实例化,实例化对象会调用类中的构造函数完成。类是一种引用数据类型,对象的实例化在堆上开辟空间。\\

\mybox{类和对象}

\begin{lstlisting}[language=Java, title=Person.java]
public class Person {
    // 属性:描述所有对象共有的特征
    public String name;
    public int age;
    public String gender;

    // 方法:描述所有对象共有的功能
    public void eat() {
        System.out.println("吃饭");
    }

    public void sleep() {
        System.out.println("睡觉");
    }
}
\end{lstlisting}

\begin{lstlisting}[language=Java, title=TestPerson.java]
public class TestPerson {
    public static void main(String[] args) {
        Person zhangsan = new Person();
        zhangsan.name = "张三";
        zhangsan.age = 18;
        zhangsan.gender = "男";

        Person lisi = new Person();
        lisi.name = "李四";
        lisi.age = 22;
        lisi.gender = "女";

        System.out.println("姓名:" + zhangsan.name 
                            + " 年龄:" + zhangsan.age
                            + " 性别:" + zhangsan.gender);
        System.out.println("姓名:" + lisi.name 
                            + " 年龄:" + lisi.age
                            + " 性别:" + lisi.gender);
        
        zhangsan.eat();
        lisi.sleep();
    }
}
\end{lstlisting}

\begin{tcolorbox}
	\mybox{运行结果}
	\begin{verbatim}
姓名:张三 年龄:18 性别:男
姓名:李四 年龄:22 性别:女
吃饭
睡觉
	\end{verbatim}
\end{tcolorbox}

\newpage

\section{封装}

\subsection{封装(Encapsulation)}

封装是面向对象方法的重要原则,就是把对象的属性和方法结合为一个独立的整体,并尽可能隐藏对象的内部实现细节。\\

封装可以认为是一个保护屏障,防止该类的数据被外部类随意访问。要访问该类的数据,必须通过严格的接口控制。合适的封装可以让代码更容易理解和维护,也加强了程序的安全性。\\

实现封装的步骤:

\begin{enumerate}
	\item 修改属性的可见性来限制对属性的访问,一般限制为private。
	\item 对每个属性提供对外的公共方法访问,也就是提供一对setter / getter,用于对私有属性的访问。
\end{enumerate}

\vspace{0.5cm}

\subsection{访问权限}

属性和方法的访问权限一般分为3种:

\begin{enumerate}
	\item public:属性和方法在类的内部和外部都可以访问。
	\item private:属性和方法只能在类内访问。
	\item protected:属性和方法只能在类的内部和其派生类中访问。
\end{enumerate}

\vspace{0.5cm}

\subsection{this指针}

每一个对象都能通过this指针来访问自身的地址,this指针是所有成员方法的隐含参数,在成员方法内部可以用来指向调用对象。\\

在类中,属性的名字可以和局部变量的名字相同。此时,如果直接使用名字来访问,优先访问的是局部变量。因此,需要使用this指针来访问当前对象的属性。\\

当需要访问的属性与局部变量没有重名的时候,this可以省略。\\

\mybox{封装}

\begin{lstlisting}[language=Java, title=Person.java]
public class Person {
    private String name;
    private int age;

    public void setName(String name) {
        this.name = name;
    }

    public void setAge(int age) {
        this.age = age;
    }

    public String getName() {
        return name;
    }

    public int getAge() {
        return age;
    }
}
\end{lstlisting}

\begin{lstlisting}[language=Java, title=TestPerson.java]
public class TestPerson {
    public static void main(String[] args) {
        Person person = new Person();
        person.setName("小灰");
        person.setAge(17);
        System.out.println("姓名:" + person.getName());
        System.out.println("年龄:" + person.getAge());
    }
}
\end{lstlisting}

\begin{tcolorbox}
	\mybox{运行结果}
	\begin{verbatim}
姓名:小灰
年龄:17
	\end{verbatim}
\end{tcolorbox}

\newpage

\section{构造方法}

\subsection{构造方法(Constructor)}

构造方法也是一个方法,用于实例化对象,在实例化对象的时候调用。一般情况下,使用构造方法是为了在实例化对象的同时,给一些属性进行初始化赋值。\\

构造方法和普通方法的区别:

\begin{enumerate}
	\item 构造方法的名字必须和类名一致。
	\item 构造方法没有返回值,返回值类型部分不写。
\end{enumerate}

如果一个类中没有写构造方法,系统会自动提供一个public权限的无参构造方法,以便实例化对象。如果一个类中已经写了构造方法,此时系统将不再提供任何默认的构造方法。\\

\mybox{构造方法}

\begin{lstlisting}[language=Java, title=Person.java]
public class Person {
    private String name;
    private int age;
    
    /**
     * 无参构造方法
     */
    public Person() {
        this.name = "";
        this.age = 0;
    }

    /**
     * 有参构造方法
     */
    public Person(String name, int age) {
        this.name = name;
        this.age = age;
    }

    public void setName(String name) {
        this.name = name;
    }

    public void setAge(int age) {
        this.age = age;
    }

    public String getName() {
        return name;
    }

    public int getAge() {
        return age;
    }
}
\end{lstlisting}

\begin{lstlisting}[language=Java, title=TestPerson.java]
public class TestPerson {
    public static void main(String[] args) {
        Person person = new Person("小灰", 17);
        System.out.println("姓名:" + person.getName());
        System.out.println("年龄:" + person.getAge());
    }
}
\end{lstlisting}

\begin{tcolorbox}
	\mybox{运行结果}
	\begin{verbatim}
姓名:小灰
年龄:17
	\end{verbatim}
\end{tcolorbox}

\newpage